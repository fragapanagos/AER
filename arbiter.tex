\documentclass[aer.tex]{subfiles}

\begin{document}

\title{Arbitration}
\maketitle
\tableofcontents

This document explores arbiters

%%%%%%%%%%%%%%%%%%%%%%%%%%%%%%%%%%%%%%%%%%%%%%%%%%%%%%%%%%%%%%%%%%%%%%%%%%%%%%%
\section{2-way Arbiter}

\begin{csp}
*[[#{C0}->C0
  \|#{C1}->C1
 ]]
\end{csp}

\begin{hse}
*[[c0i->c0o+;[~c0i];c0o-
  \|c1i->c1o+;[~c1i];c1o-
 ]]
\end{hse}

\begin{hse}
*[[c0i->c0+;[~c0i];c0-
  \|c1i->c1+;[~c1i];c1-
 ]]

*[[c0->c0o+;[~c0];c0o-
  []c1->c1o+;[~c1];c1o-
 ]]
\end{hse}

Cross-coupled NANDs

\begin{prs2}
c0i & _c1 -> _c0-
~c0i | ~_c1 -> _c0+

c1i & _c0 -> _c1-
~c1i | ~_c0 -> _c1+
\end{prs2}

Filter circuit to ensure $\_c0$ and $\_c1$ are separated by at least a PMOS's threshold voltage

\begin{prs2}
_c1 & ~_c0 -> c0o+
~_c1 | _c0 -> c0o-

_c0 & ~_c1 -> c1o+
~_c0 | _c1 -> c1o-
\end{prs2}

%%%%%%%%%%%%%%%%%%%%%%%%%%%%%%%%%%%%%%%%%%%%%%%%%%%%%%%%%%%%%%%%%%%%%%%%%%%%%%%
\section{N-way Arbiter (N\_ARB) Specification}

The standard arbiter arbitrates between 2 clients and is comprised of 2 cross-coupled NAND gates and a filter circuit for a total of 12 transistors.
In this document, we develop arbiters for N clients.

The N-way arbiter $N\_ARB$ arbitrates between N clients (see Figure~\ref{fig:n_arb}) and is specified by

\begin{figure}
  \centering
  \includegraphics[width=.5\textwidth]{img/transmitter/n_arb.pdf}
  \caption{N\_ARB arbitrates between N children.}
  \label{fig:n_arb}
\end{figure}

\begin{csp}
N_ARB\equiv
  *[[\langle\|n:0..N\-1:#{C`n}->C`n\rangle]]
\end{csp}

\begin{figure}
  \centering
  \includegraphics[width=.5\textwidth]{img/transmitter/n_arb_n_arb_.pdf}
  \caption{N\_ARB decomposed as a binary tree of N\_ARB\_ cells. Here is an N\_ARB instance servicing 4 children. The top N\_ARB\_ cell can simplified to not have a P port or have its P connected back to itself.}
  \label{fig:n_arb_n_arb_}
\end{figure}

We construct the N\_ARB as a binary tree of N\_ARB\_ cells recursively (see Figure~\ref{fig:n_arb_n_arb_}). 

%%%%%%%%%%%%%%%%%%%%%%%%%%%%%%%%%%%%%%%%%%%%%%%%%%%%%%%%%%%%%%%%%%%%%%%%%%%%%%%
\section{Unpipelined N-way Arbiter}

This is the arbiter in Rajit's notes to provide mutual exclusion to a shared resource. 
We extend the development with bubble-reshuffled, CMOS-implementable PRS and transistor counts.

\subsection{CHP}

We implement this process as a tree of nodes that grant mutually exclusive access to a shared resource.

\begin{csp}
N_ARB_\equiv
*[[#{C`0}->P;C`0
  \|#{C`1}->P;C`1
 ]]
\end{csp}

Client requests are directed up the tree and fulfilled once the parent nodes grant permission. 
There is no pipelining in this design.

We decompose this into concurrent mutual exclusion (MU\_EX) and parent requesting (PREQ) processes.

\begin{csp}
MU_EX\equiv
*[[#{C`0}->Q`0;C`0
  \|#{C`1}->Q`1;C`1
 ]]
\end{csp}
\begin{csp}
PREQ\equiv
*[[#{Q`0}->P;Q`0
  []#{Q`1}->P;Q`1
 ]]
\end{csp}

\subsection{HSE}

MU\_EX expands to
\begin{hse}
*[[c0i->q0o+;[q0i];c0o+;[~c0i];q0o-;[~q0i];c0o-
  \|c1i->q1o+;[q1i];c1o+;[~c1i];q1o-;[~q1i];c1o-
 ]]
\end{hse}

We break out the arbitration.

\begin{hse}
*[[c0i->a0+;[~c0i];a0-
  \|c1i->a1+;[~c1i];a1-
 ]]
*[[a0->q0o+;[q0i];c0o+;[~a0];q0o-;[~q0i];c0o-
  []a1->q1o+;[q1i];c1o+;[~a1];q1o-;[~q1i];c1o-
 ]]
\end{hse}

PREQ expands to

\begin{hse}
*[[q0i->po+;[pi];q0o+;[~q0i];po-;[~pi];q0o-
  []q1i->po+;[pi];q1o+;[~q1i];po-;[~pi];q1o-
 ]]
\end{hse}

\subsection{PRS}

MU\_EX:

\begin{prs2}
~c1o & a0 -> q0o+
c1o | ~a0 -> q0o-

~c0o & a1 -> q1o+
c0o | ~a1 -> q1o-
\end{prs2}

\begin{prs2}
q0i -> c0o+
~q0i -> c0o-

q1i -> c1o+
~q1i -> c1o-
\end{prs2}

PREQ:

\begin{prs2}
q0i | q1i -> po+
~q0i & ~q1i -> po-
\end{prs2}

\begin{prs2}
pi & q0i -> q0o+
~pi -> q0o-

pi & q1i -> q1o+
~pi -> q1o-
\end{prs2}

\subsection{CMOS-implementable PRS}

MU\_EX:

\begin{prs2}
_c1o & a0 -> _q0o-
~_c1o | ~a0 -> _q0o+

_c0o & a1 -> _q1o-
~_c0o | ~a1 -> _q1o+
\end{prs2}

\begin{prs2}
q0i -> _c0o-
~q0i -> _c0o+

q1i -> _c1o-
~q1i -> _c1o+
\end{prs2}

accounting:

\begin{center}
    \begin{tabular}{|r|l|l|}
    \hline
    rule & transistor count & comments \\ \hline
    $a[0,1]$ & 12 & 2-way arbiter \\ \hline
    $\_q[0,1]_o$ & 8 & \\ \hline
    $\_c[0,1]_o$ & 4 & staticizes PREQ $\_q[0,1]_o$ \\ \hline
    \hline total & 24 & \\ \hline
    \end{tabular}
\end{center}

PREQ:

\begin{prs2}
~_q0i | ~_q1i -> po+
_q0i & _q1i -> po-
\end{prs2}

\begin{prs2}
~_pi & ~_q0i -> q0o+
_pi -> q0o-

~_pi & ~_q1i -> q1o+
_pi -> q1o-
\end{prs2}

accounting:

\begin{center}
    \begin{tabular}{|r|l|l|}
    \hline
    rule & transistor count & comments \\ \hline
    $p_o$ & 4 & \\ \hline
    $\_q[0,1]_o$ & 10 & staticizers placed on MU\_EX $\_c[0,1]_o$ rules \\ \hline
    \hline total & 14 & \\ \hline
    \end{tabular}
\end{center}

\subsection{Accounting}

\begin{center}
    \begin{tabular}{|r|l|l|l|}
    \hline
    component & transistors/component & components/node & transistors/node \\ \hline
    MU\_EX & 24 & 1 & 24 \\ \hline
    PREQ & 14& 1 & 14 \\ \hline
    \multicolumn{3}{|r|}{total transistors/node} & 38 \\ \hline
    \end{tabular}
\end{center}

N\_ARB\_ is constructed as a tree with a 2-way arbiter at the root and N-2 instances of N\_ARB\_. 
Overall, N\_ARB costs $38(N-2)+12$ transistors.

\begin{center}
  \begin{tabular}{|r|c|c|c|c|c|c|c|c|c|}
    \hline
    N & 2 & 3 & 4 & 8 & 16 & 32 & 64 & 128 & 256 \\
    \hline
    transistors & 12 & 50 & 88 & 240 & 544 & 1152 & 2368 & 4800 & 9664 \\
    \hline
  \end{tabular}
\end{center}

If the clients expect active high inputs, we'll need N inverters to convert the output of N\_ARB\_ to active high.
In this case, N\_ARB costs $38(N-2)+12+2N$ transistors for $N>2$.

\begin{center}
  \begin{tabular}{|r|c|c|c|c|c|c|c|c|c|}
    \hline
    N & 2 & 3 & 4 & 8 & 16 & 32 & 64 & 128 & 256 \\
    \hline
    transistors & 12 & 56 & 96 & 256 & 576 & 1216 & 2496 & 5056 & 10176 \\
    \hline
  \end{tabular}
\end{center}

Note that this implementation does not gaurantee mutual exclusion of the output signals between
clients across leaves. For example, consider a 4-input arbiter with inputs 0 and 3 high. The N-way arbiter selects 0 first. 0 clears its input. The parent arbiter node clears $p_i$, and the 0 output will clear sometime in the future. However, because $p_i$ has cleared, the parent node is free to select the other child and can complete the full communication with 3 even before the 0 output clears. To fix this, the outputs would have to mutually exclude the requests of all other clients across the leaf nodes.

%%%%%%%%%%%%%%%%%%%%%%%%%%%%%%%%%%%%%%%%%%%%%%%%%%%%%%%%%%%%%%%%%%%%%%%%%%%%%%%
\section{Alternative Unpipelined N-way Arbiter}

Here's an alternative to the previous, unpipelined design.

\subsection{CHP}

\begin{csp}
N_ARB_\equiv
  *[[C0\star\!P;C0\star\!P
    \|C1\star\!P;C1\star\!P]]
\end{csp}

\subsection{HSE}

\begin{hse}
*[[c0i->c0+;[~c0i];c0-
  \|c1i->c1+;[~c1i];c1-]]
*[[c0->po+;[pi];c0o+;[~c0];po-;[~pi];c0o-
  []c1->po+;[pi];c1o+;[~c1];po-;[~pi];c1o-]]
\end{hse}

\subsection{PRS}

\begin{prs2}
c0 & ~c1o | c1 & ~c0o -> po+
~c0 & c0o | ~c1 & c1o -> po-
\end{prs2}

\begin{prs2}
pi & c0 & ~c1o -> c0o+
~pi -> c0o-

pi & c1 & ~c0o -> c1o+
~pi -> c1o-
\end{prs2}

\subsection{CMOS-implementable PRS}

\noindent A standard 2-input arbiter takes in $c0_i$ and $c1_i$ to generate $c0$ and $c1$.

\begin{prs2}
c0 & _c1o | c1 & _c0o -> _po-
~c0 & ~_c0o | ~c1 & ~_c1o -> _po+
\end{prs2}

\begin{prs2}
pi & c0 & _c1o -> _c0o-
~pi -> _c0o+

pi & c1 & _c0o -> _c1o-
~pi -> _c1o+
\end{prs2}

\begin{prs2}
_po -> po-
~_po -> po+

_c0o -> c0o-
~_c0o -> c0o+

_c1o -> c1o-
~_c1o -> c1o+
\end{prs2}

accounting:

\begin{center}
    \begin{tabular}{|r|l|l|}
    \hline
    rule & transistor count & comments \\ \hline
    $\_p_o$ & 8 & \\ \hline
    $c[0,1]$ & 12 & standard 2-way arbiter \\ \hline
    $\_c[0,1]_o$ & 8 & \\ \hline
    $p_o$ & 4 & staticizer \\ \hline
    $c[0,1]_o$ & 8 & staticizer \\ \hline
    \hline total & 40 & \\ \hline
    \end{tabular}
\end{center}

\subsection{Accounting}

N\_ARB\_ is constructed as a tree with a 2-way arbiter at the root and $N-2$ instances of N\_ARB\_. Overall, N\_ARB\ costs $40(N-2)+12$ transistors (See Table~\ref{tab:cheap_n_arb_cost}).

\begin{table}[ht]
  \centering
  \begin{tabular}{|r|c|c|c|c|c|c|c|c|c|}
    \hline
    N & 2 & 3 & 4 & 8 & 16 & 32 & 64 & 128 & 256 \\
    \hline
    transistors & 12 & 52 & 92 & 252 & 572 & 1212 & 2492 & 5052 & 10172 \\
    \hline
  \end{tabular}
  \caption{\label{tab:cheap_n_arb_cost}Transistor count for common sizes of $N\_ARB\_C$}
\end{table}

%%%%%%%%%%%%%%%%%%%%%%%%%%%%%%%%%%%%%%%%%%%%%%%%%%%%%%%%%%%%%%%%%%%%%%%%%%%%%%%
\section{Pipelined N-way Arbiter}

In this design, 

\begin{csp}
N\_ARB\_ \equiv H\_ARB
\end{csp}
%%%%%%%%%%%%%%%%%%%%%%%%%%%%%%%%%%%%
\subsection{Heirarchical arbiter H\_ARB}

H\_ARB coordinates with its parent (another H\_ARB instance) and arbitrates between two children. 
With ports

\begin{tabular}[]{rl}
$C0$ & child 0 port \\
$C1$ & child 1 port \\
$P$ & parent port \\
\end{tabular} \\

\noindent H\_ARB is specified by

\begin{csp}
H_ARB\equiv
  *[[#{C`0}|#{C`1}->P;
    [#{C`0}->C`0;C`0\|~#{C`0}->skip];
    [#{C`1}->C`1;C`1\|~#{C`1}->skip];P]]
\end{csp}

When a child request arrives, H\_ARB sends a request to its parent. When the parent gives permission to go ahead, H\_ARB checks the first child and then the second child. During each check, H\_ARB services the child if its request is present. Otherwise, H\_ARB skips and moves on. After checking the children, H\_ARB releases its parent.

H\_ARB implements a greedy yet fair arbitration algorithm. H\_ARB is greedy because it can service both children for a single request to the parent. H\_ARB is fair because it will service each child at most once for each request to the parent. 

\begin{figure}
  \centering
  \includegraphics[width=.5\textwidth]{img/transmitter/arb_h_detail.pdf}
  \caption{$H\_ARB$ decomposition. Dotted line indicates wire used for probing only.}
  \label{fig:h_arb_detail}
\end{figure}

We decompose H\_ARB into a control cell $CTRL$ and two child arbitration cells $C\_ARB$ as in Figure~\ref{fig:h_arb_detail}.

\begin{csp}
H\_ARB\equiv\!CTRL\pll\!C_ARB\pll\!C_ARB
\end{csp}

\begin{figure}
  \centering
  \includegraphics[width=.7\textwidth]{img/transmitter/H_ARB_bubble_reshuffled.pdf}
  \caption{H\_ARB after bubble reshuffling. Ports are expanded and inverters introduced by bubble reshuffling are shown.}
  \label{fig:h_arb_bubbled}
\end{figure}

\noindent Bubble reshuffling $CTRL$ and $C\_ARB$ yields the H\_ARB shown in Figure~\ref{fig:h_arb_bubbled}.

%%%%%%%%%%%%%%%%%%%%%%%%%%%%%%%%%%%%%%%
\subsection{Control cell CTRL}
CTRL sequences communications with the parent of H\_ARB and the child arbiters.
CTRL has ports

\begin{tabular}[]{rl}
$P$ & parent port \\
$S_0$ & child arbiter 0 port \\
$S_1$ & child arbiter 1 port \\
\end{tabular} \\

\subsubsection{CHP}

\begin{csp}
CTRL\equiv
  *[[#{C`0}|#{C`1}->P;S`0;S`0;S`1;S`1;P]]
\end{csp}

\noindent where 

\begin{tabular}[]{rl}
  $C_0$ & probes child 0 \\
  $C_1$ & probes child 1 \\
  $P$ & communicates with the parent \\
  $S_0$ & communicates with the child 0 arbiter \\
  $S_1$ & communicates with the child 1 arbiter \\
\end{tabular} \\ \\

\subsubsection{HSE}

\noindent Directly translating the CHP,

\begin{hse}
CTRL\equiv
  *[[c0i|c1i];po+;[pi];
    s0o+;[s0i];[~s0i];s0o-;
    s1o+;[s1i];[~s1i];s1o-;
    po-;[~pi]]
\end{hse}

\noindent There are 3 indistinguishable states--after $[p_i]$, after $s0_o\!\downarrow$, and after $s1_o\!\downarrow$. The correct operation of CTRL precludes reshuffling to break symmetry, so we use 2 state variables to distinguish betwen the 3 states.

\begin{hse}
CTRL\equiv
  *[[c0i|c1i];po+;[pi];x-;
    s0o+;[s0i];y-;s0o-;[~s0i];
    s1o+;[s1i];x+;s1o-;[~s1i];
    po-;[~pi];y+]
\end{hse}

\subsubsection{PRS}

\begin{prs2}
(c0i | c1i) & y -> po+
~s1i & x & ~y -> po-
\end{prs2}

\begin{prs2}
s1i & ~y -> x+
pi & y -> x-

~pi -> y+
s0i -> y-
\end{prs2}

\begin{prs2}
~x & y -> s0o+
x | ~y -> s0o-

~s0i & ~x & ~y -> s1o+
x -> s1o-
\end{prs2}

\noindent after bubble reshuffling

\begin{prs2}
(~_c0i | ~_c1i) & ~_y -> po+
_s1i & x & _y -> po-

po -> _po-
~po -> _po+
\end{prs2}

\begin{prs2}
~x & ~_y -> s0o+
x | _y -> s0o-

~s0i & ~x & ~y -> s1o+
x -> s1o-
\end{prs2}

\begin{prs2}
~_s1i & ~y -> x+
pi & y -> x-

~pi -> y+
s0i -> y-
\end{prs2}

Including staticizers, CTRL contains 36 transistors.

%%%%%%%%%%%%%%%%%%%%%%%%%%%%%%%%%%%%%%%
\subsection{Child Arbiter (C\_ARB)}
C\_ARB determines whether or not a child's request is present when called upon by CTRL.

\subsubsection{CHP}

\begin{csp}
C_ARB\equiv
  *[[#C->S;C;C;S
    \|#S->S;S]]
\end{csp}

\noindent where 

\begin{tabular}[]{rl}
  $C$ & communicates with the child \\ 
  $S$ & communicates with $CTRL$ \\
\end{tabular} \\ \\

We arbitrate between requests from the child, $\overline{C}$, and 
requests from CTRL, $\overline{S}$. If a child request is present, the arbiter is biased towards selecting the child because $S$ requests will not arrive until CTRL has obtained permission from the parent. Although small, there is a chance that a child's request will be skipped on a given iteration of CTRL, but the weakly fair assumption ensures that the child will be serviced eventually.

\noindent If the arbitration selects the child request $\overline{C}$, C\_ARB

\begin{tabular}[]{rl}
  $S$ & waits for and acknowledge CTRL's request \\
  $C$ & signals the child to proceed \\
  $C$ & waits for the child to complete \\
  $S$ & releases CTRL so it can move on \\
\end{tabular} \\ \\

\noindent If the arbitration selects CTRL's request $\overline{S}$, C\_ARB

\begin{tabular}[]{rl}
  $S$ & waits for and acknowledge CTRL's request \\
  $S$ & releases CTRL so it can move on \\
\end{tabular} \\ \\

\noindent which is a simple 4-phase handshake. Note that $S$ is shared between the two branches.

\subsubsection{HSE}

\noindent Directly translating the CHP,

\begin{hse}
*[[ci->[si];so+;co+;[~ci];co-;[~si];so-
  \|si->so+;[~si];so-]]
\end{hse}

\noindent We introduce intermediate variables $c$ and $s$ to represent the arbitration,

\begin{hse}
*[[ci->c+;[si];so+;co+;[~ci];c-;co-;[~si];so-
  \|si->s+;so+;[~si];s-;so-]]
\end{hse}

\noindent and break out the arbitration into a separate process

\begin{hse}
*[[ci->c+;[~ci];c-
  \|si->s+;[~si];s-]]
*[[c->[si];so+;co+;[~c];co-;[~si];so-
  []s->so+;[~s];so-]]
\end{hse}

\noindent Note that the $c$ branch of the select statement uses $s_i$ and not $s$. In addition, as soon as $c_o$ goes high, $c_i$ could clear before $s_i$, which would allow the arbitration process to complete its $c_i$ branch and select the $s_i$ branch while the $c$ branch is still in progress. We thus wait for $s_i$ to clear before communicating with the child.

\begin{hse}
*[[ci->c+;[~ci];c-
  \|si->s+;[~si];s-]]
*[[c->[si];so+;[~si];co+;[~c];co-;so-
  []s->so+;[~s];so-]]
\end{hse}

\noindent After $co\!\downarrow$, the child could raise its request again ($c_i$) and greedily hold the circuit in a loop servicing the child. We break this loop by releasing CTRL ($s_o\!\downarrow$) before releasing the child.

\begin{hse}
*[[ci->c+;[~ci];c-
  \|si->s+;[~si];s-]]
*[[c->[si];so+;[~si];co+;[~c];so-;co-
  []s->so+;[~s];so-]]
\end{hse}

\noindent Note that $s_o$ is shared between both branches of selection statement. This sharing results in interfering production rules for $s_o$, so we introduce intermediate values $cs_o$ and $ss_o$ and merge them in a third concurrent process.

\begin{hse}
*[[ci->c+;[~ci];c-
  \|si->s+;[~si];s-]]\pll
*[[c->[si];cso+;[~si];co+;[~c];cso-;co-
  []s->sso+;[~s];sso-]]\pll
*[[cso->so+;[~cso];so-]
  []sso->so+;[~sso];so-]
\end{hse}

\subsubsection{PRS}

\noindent $s_i$ and $c_i$ are inputs to a standard 2-input arbiter, which outputs $s$ and $c$.

\begin{prs2}
c & si & ~sso -> cso+
~c -> cso-

s & ~co -> sso+
~s | co -> sso-
\end{prs2}

\begin{prs2}
~si & cso -> co+
si | ~cso -> co-

cso | sso -> so+
~cso & ~sso -> so-
\end{prs2}

\noindent after bubble reshuffling

\begin{prs2}
~_c & ~_si & ~sso -> cso+
_c -> cso-

~_s & ~co -> sso+
_s | co -> sso-
\end{prs2}

\begin{prs2}
_si & cso -> _co-
~_si | ~cso -> _co+

_co -> co-
~_co -> co+

cso | sso -> _so-
~cso & ~sso -> _so+
\end{prs2}

\noindent where $\_s$ and $\_c$ come from an active-low 2-input arbiter with inputs $\_s_i$ and $\_c_i$. 

\noindent Including staticizers, C\_ARB requires 34 transistors.

%%%%%%%%%%%%%%%%
\subsection{H\_ARB Accounting}

H\_ARB includes 2 C\_ARB circuits and 1 CTRL circuit with 3 additional inverters. Therefore, H\_ARB requires 114 transistors.

%%%%%%%%%%%%%%%%%%%%%%%%%%%%%%%%%%%%
\subsection{Accounting}

N\_ARB will be constructed from $N-1$ H\_ARB instances. We will also require N inverters to connect with the client's input and 1 inverter to connect the top of the tree back to itself. Therefore, N\_ARB requires $110(N-1)+2N+2=112(N-1)+4$ transistors.

\begin{center}
  \begin{tabular}{|r|c|c|c|c|c|c|c|c|c|}
    \hline
    N & 2 & 3 & 4 & 8 & 16 & 32 & 64 & 128 & 256 \\
    \hline
    transistors & 116 & 228 & 340 & 788 & 1684 & 3476 & 7060 & 14228 & 28564 \\
    \hline
  \end{tabular}
\end{center}

Note that the basic, 2-input arbiter only requires 12 transistors, tenfold less than a 2-input N\_ARB.

%%%%%%%%%%%%%%%%%%%%%%%%%%%%%%%%%%%%%%%%%%%%%%%%%%%%%%%%%%%%%%%%%%%%%%%%%%%%%%%
\end{document}
