\documentclass{article}
\usepackage{mystyle}

\begin{document}
\title{Serialized Tree AER}
\author{Sam Fok}
\maketitle

The neuron address packet is serialized as a series of 1-of-N words. 
We use an additional line (making the total data lines 1-of-(N+1)) to indicate the completion of a packet.

%%%%%%%%%%%%%%%%%%%%%%%%%%%%%%%%%%%%%%%%%%%%%%%%%%%%%%%%%%%%%%%%%%%%%%%%%%%%%%%
\section{Transmitter ($AEXT$)}

The transmitter is organized as a tree. Packets travel from the leaves of the tree to the root. At each node, incoming packet streams are merged into a single output stream. Each packet is prepended with a word indicating which branch it came from. To merge the streams, incoming packets from different branches are output one after the other. 

\begin{csp}
NODE\equiv
*[[h->
    [#{C0}->s:=0,P!(0);
    \|#{C1}->s:=1,P!(1)];
    h:=false
  []~h->
    [s=0->C0?x;P!x
    []s=1->C1?x;P!x
    ];h:=x.tail
 ]]
\end{csp}

We divide NODE into PFWD and MERGE processes. 
PFWD prepends a word to the packet indicating which branch the packet is coming from.
MERGE arbitrates between incoming packet streams and outputs them one at a time.

%%%%%%%%%%%%%%%%%%%%%%%%%%%%%%%%%%%
\subsection{PFWD version hq}

This version has fewer state variables than version 2, but the pull-up and pull-down chains are too long.

%%%%%%%%%%%%%%%
\subsubsection*{CHP}

\begin{csp}
PFWD\equiv
  h:=true;
  *[[h&#{X}->Y!(\textrm{header});h-
    []~h&#{X}->Y!(X?)\*[X=t->h+];
    ]
   ]
\end{csp}

%%%%%%%%%%%%%%%
\subsubsection*{HSE}

\begin{hse}
h:=true;
*[h&(x0|x1|xt)->yn+;[yi];q+;yn-;[~yi];h-;q-
  []~h&x0->y0+;xo+;[yi];y0-;[~x0];xo-;[~yi]
  []~h&x1->y1+;xo+;[yi];y1-;[~x1];xo-;[~yi]
  []~h&xt->yt+;xo+;[yi];h+;yt-;[~xt];xo-;[~yi]
 ]
\end{hse}

%%%%%%%%%%%%%%%
\subsubsection*{PRS}

\begin{prs2}
h & yi & yn -> q+
~h -> q-

yt & xo & yi -> h+
q & ~yi -> h-
\end{prs2}

\begin{prs2}
~h & (y0 | y1 | yt) -> xo+
~x0 & ~x1 & ~xt & ~y0 & ~y1 & ~yt -> xo-
\end{prs2}

\begin{prs2}
h & ~q & ~yi & ~xo & (x0 | xi | xt) -> yn+
q -> yn-
\end{prs2}

\begin{prs2}
~h & ~q & ~yi & x0 & ~xo -> y0+
yi & xo -> y0-

~h & ~q & ~yi & x1 & ~xo -> y1+
yi & xo -> y1-
\end{prs2}

\begin{prs2}
~h & ~q & ~yi & xt & ~xo -> yt+
h & xo -> yt-
\end{prs2}

%%%%%%%%%%%%%%%%%%%%%%%%%%%%%%%%%%%
\subsection{PFWD version hu}

This version has more state variables than version 1, but has reasonable pull-up and pull-down chains.

%%%%%%%%%%%%%%%
\subsubsection*{CHP}

\begin{csp}
PFWD\equiv
  *[[h&#{X}->Y!(\textrm{header});h-
    []~h&#{X}->X?u\*Y!u,[u=t->h+]
    ]
   ]
\end{csp}

%%%%%%%%%%%%%%%
\subsubsection*{HSE}

\begin{hse}
*[[h&(x0|x1|xt)->yn+;[yi];h-;yn-;[~yi]
  []~h&x0->u0+;(xo+;[~x0]),(y0+;[yi]);u0-;(y0-;[~yi]),xo-
  []~h&x1->u1+;(xo+;[~x1]),(y1+;[yi]);u1-;(y1-;[~yi]),xo-
  []~h&xt->ut+;(xo+;[~xt]),(yt+;h+;[yi]);ut-;(yt-;[~yi]),xo-
  ]
 ]
\end{hse}

\begin{hse}
*[[h&(x0|x1|xt)->yn+;[yi];h-;yn-;[~yi]
  []~h&x0->u0+;[~x0&yi];u0-;[~yi]
  []~h&x1->u1+;[~x1&yi];u1-;[~yi]
  []~h&xt->ut+;[~xt&h&yi];ut-;[~yi]
  ]
 ]

*[u0->xo+,y0+;[~u0];y0-,xo-
  []u1->xo+,y1+;[~u1];y1-,xo-
  []ut->xo+,(yt+;h+);[ut-];yt-,xo-
 ]
\end{hse}

%%%%%%%%%%%%%%%
\subsubsection*{PRS}

\begin{prs2}
yt -> h+
yn & yi -> h-
\end{prs2}

\begin{prs2}
h & (x0 | x1 | xt) & ~yi & ~yt -> yn+
~h & yi & ~un -> yn-
\end{prs2}

\begin{prs2}
u0 | u1 | ut -> xo+
~u0 & ~u1 & ~ut -> xo-
\end{prs2}

\begin{prs2}
~h & x0 & ~yi -> u0+
~x0 & yi -> u0-

~h & x1 & ~yi -> u1+
~x1 & yi -> u1-
\end{prs2}

\begin{prs2}
~h & xt & ~yi -> ut+
h & ~xt & yi -> ut-
\end{prs2}

\begin{prs2}
u0 -> y0+
~h & ~u0 -> y0-

u1 -> y1+
~h & ~u1 -> y1-
\end{prs2}

\begin{prs2}
ut -> yt+
~ut -> yt-
\end{prs2}

%%%%%%%%%%%%%%%%%%%%%%%%%%%%%%%%%%%
\subsection{Leaf interface LEAF\_INT}

LEAF\_INT interfaces with the neuron and outputs a tail word to get the party started.

%%%%%%%%%%%%%%%
\subsubsection*{CHP}
\begin{csp}
LEAF_INT\equiv
  *[C\star\!P!\textrm{tail};C\star\!P!\textrm{tail}]
\end{csp}

%%%%%%%%%%%%%%%
\subsubsection*{HSE}
\begin{hse}
*[[ci];{po}`t+;[pi]co+;
  [~ci];{po}`t-;[~pi];co-]
\end{hse}

%%%%%%%%%%%%%%%
\subsubsection*{PRS}
\begin{prs2}
ci -> {po}`t+
~ci -> {po}`t-

pi -> co+
~pi -> co-
\end{prs2}

\noindent These are just wires...

%%%%%%%%%%%%%%%%%%%%%%%%%%%%%%%%%%%
\subsection{MERGE}

MERGE sequences between outputting two serialized packet streams.

\begin{csp}
MERGE\equiv
  *[[~l&~r->
      [#{L}->l:=true
      \|#{R}->r:=true
      ]
    []l|r->
      [l->O!(L?)
      []r->O!(R?)
    ]
  ]]
\end{csp}

%%%%%%%%%%%%%%%%%%%%%%%%%%%%%%%%%%%%
\subsection{MERGE version a\_a}

I don't like this version because the pullup chains for the state variables
are too long and won't scale to higher radix encoding.

%%%%%%%%%%%%%%%
\subsubsection*{CHP}

\begin{csp}
MERGE\equiv
  *[[h->[#{C0}->a:=0\|#{C1}->a:=1];h-
    []~h&a=0->P!(C0?)
    []~h&a=0->P!(C1?)
    ]]
\end{csp}

%%%%%%%%%%%%%%%
\subsubsection*{HSE}

\begin{hse}
*[[~ao&~a1&(c00|c01|c0t)->a0+
  \|~a0&~a1&(c10|c11|c1t)->a1+]]

*[[ a0&c00->p0+;c0o+;[pi&~c00];p0-;c0o-;[~pi]
  []a0&c01->p1+;c0o+;[pi&~c01];p1-;c0o-;[~pi]
  []a0&c0t->pt+;c0o+;[pi&~c0t];a0-;pt-;c0o-;[~pi]
  []a1&c10->p0+;c1o+;[pi&~c10];p0-;c1o-;[~pi]
  []a1&c11->p1+;c1o+;[pi&~c11];p1-;c1o-;[~pi]
  []a1&c1t->pt+;c1o+;[pi&~c1t];a1-;pt-;c1o-;[~pi]
  ]]
\end{hse}
%%%%%%%%%%%%%%%
\subsubsection*{PRS}

\begin{prs2}
~a0 & (c00 | c01| c0t) & ~c0o -> a0i+
a0 -> a0i-

~a1 & (c10 | c11| c1t) & ~c1o -> a1i+
a1 -> a1i-

a0o & ~a1 & ~c1o & ~pi -> a0+ % uhoh will depend on other a's
~a0o & pi & pt & ~c0t -> a0-

a1o & ~a0 & ~c0o & ~pi -> a1+
~a1o & pi & pt & ~c1t -> a1-
\end{prs2}

\begin{prs2}
~pi & (a0 & c00 | a1 & c10) -> p0+
pi & (~_a0 & ~c00 | ~_a1 & ~c10) -> p0-

~pi & (a0 & c01 | a1 & c11) -> p1+
pi & (~_a0 & ~c01 | ~_a1 & ~c11) -> p1-

~pi & (a0 & c0t | a1 & c1t) -> pt+
_a0 & _a1 -> pt-
\end{prs2}

\begin{prs2}
a0 & (p0 | p1 | pt) -> c0o+
~p0 & ~p1 & ~pt -> c0o-

a1 & (p0 | p1 | pt) -> c1o+
~p0 & ~p1 & ~pt -> c1o-
\end{prs2}

%%%%%%%%%%%%%%%%%%%%%%%%%%%%%%%%%%%%
\subsection{MERGE version ah}

This one has acceptable pullup/pulldown chains, but I'm worried about making it CMOS implementable

%%%%%%%%%%%%%%%
\subsubsection*{HSE}

\begin{hse}
*[[~a0&(c00|c01|c0t)->a0+;h-;[~a0];h+
  \|~a1&(c10|c11|c1t)->a1+;h-;[~a1];h+]]

*[[ a0&c00->p0+;c0o+;[pi&~c00];p0-;c0o-;[~pi]
  []a0&c01->p1+;c0o+;[pi&~c01];p1-;c0o-;[~pi]
  []a0&c0t->pt+;c0o+;[pi&~c0t];a0-;pt-;c0o-;[~pi]
  []a1&c10->p0+;c1o+;[pi&~c10];p0-;c1o-;[~pi]
  []a1&c11->p1+;c1o+;[pi&~c11];p1-;c1o-;[~pi]
  []a1&c1t->pt+;c1o+;[pi&~c1t];a1-;pt-;c1o-;[~pi]
  ]]
\end{hse}

%%%%%%%%%%%%%%%
\subsubsection*{PRS}

\begin{prs2}
(c00 | c01| c0t) & ~a0 -> a0i+
~h & a0 -> a0i-

(c10 | c11| c1t) & ~a1-> a1i+
~h & a1 -> a1i-

h & a0o & ~pi -> a0+
pt & pi & c0o & ~c0t & ~a0o -> a0-

h & a1o & ~pi -> a1+
pt & pi & c1o & ~c1t & ~a1o -> a1-
\end{prs2}

\begin{prs2}
~a0 & ~a1 -> h+
a0 | a1 & ~c0o & ~c1o-> h-
\end{prs2}

\begin{prs2}
~pi & (a0 & c00 | a1 & c10) & ~h -> p0+
pi & (a0 & ~c00 | a1 & ~c10) -> p0-

~pi & (a0 & c01 | a1 & c11) & ~h -> p1+
pi & (a0 & ~c01 | a1 & ~c11) -> p1-

~pi & (a0 & c0t | a1 & c1t) & ~h -> pt+
~a0 & ~a1 -> pt-
\end{prs2}

\begin{prs2}
a0 & (p0 | p1 | pt) -> c0o+
~p0 & ~p1 & ~pt -> c0o-

a1 & (p0 | p1 | pt) -> c1o+
~p0 & ~p1 & ~pt -> c1o-
\end{prs2}

%%%%%%%%%%%%%%%%%%%%%%%%%%%%%%%%%%%%%%%%%%%%%%%%%%%%%%%%%%%%%%%%%%%%%%%%%%%%%%%
\section{Receiver ($AERV$)}

We'll decompose the node into ROUTE, READ\_HEAD, and FWD\_BODY.


%%%%%%%%%%%%%%%%%%%%%%%%%%%%%%%%%%%%%%%%%%%%%%%%%%%%%%%%%%%%%%%%%%%%%%%%%%%%%%%
\section{ROUTE}

ROUTE sends a parent's signal to one of its children depending on which child requests. Assumes requests are mutually exclusive.

%%%%%%%%%%%%%%%%%%%%%%%%%%%%%%%%%%%%%%%%%%%%%%%%%%%%%%%%%%%%
\subsection{ROUTE\_unpipelined}

%%%%%%%%%%%%%%%%%%%%%%%%%%%%%%%%%%%%%%%%
\subsubsection*{CHP}

\begin{csp}
*[[#{C0!};C0!(P?)
  []#{C1!};C1!(P?)]
 ]
\end{csp}

%%%%%%%%%%%%%%%%%%%%%%%%%%%%%%%%%%%%%%%%
\subsubsection*{HSE}

% [[ci];pe+;[pi];co+;[~ci];pe-;[~pi];co-]
\begin{hse}
*[[c0e|c1e];pe+
    [ p0&c0e->c00+;[~c0e];pe-;[~p0];c00-
    []p1&c0e->c01+;[~c0e];pe-;[~p1];c01-
    []pt&c0e->c0t+;[~c0e];pe-;[~pt];c0t-
    []p0&c1e->c10+;[~c1e];pe-;[~p0];c10-
    []p1&c1e->c11+;[~c1e];pe-;[~p1];c11-
    []pt&c1e->c1t+;[~c1e];pe-;[~pt];c1t-
    ]
 ]
\end{hse}

%%%%%%%%%%%%%%%%%%%%%%%%%%%%%%%%%%%%%%%%
\subsubsection*{PRS}

\begin{prs2}
c0e | c1e -> pe+
~c0e & ~c1e -> pe-
\end{prs2}

\begin{prs2}
p0 & c0e -> c00+
~p0 -> c00-

p1 & c0e -> c01+
~p1 -> c01-

pt & c0e -> c0t+
~pt -> c0t-

p0 & c1e -> c10+
~p0 -> c10-

p1 & c1e -> c11+
~p1 -> c11-

pt & c1e -> c1t+
~pt -> c1t-
\end{prs2}

%%%%%%%%%%%%%%%%%%%%%%%%%%%%%%%%%%%%%%%%%%%%%%%%%%%%%%%%%%%%
\subsection{ROUTE\_pipelined}

%%%%%%%%%%%%%%%%%%%%%%%%%%%%%%%%%%%%%%%%
\subsubsection*{HSE}

% [pe+;[pi&ci];co+;pe-;[~pi&~ci];co-] % P;C;P;C
\begin{hse}
*[pe+;
    [ p0&c0e->c00+;pe-;[~p0&~c0e];c00-
    []p1&c0e->c01+;pe-;[~p1&~c0e];c01-
    []pt&c0e->c0t+;pe-;[~pt&~c0e];c0t-
    []p0&c1e->c10+;pe-;[~p0&~c0e];c10-
    []p1&c1e->c11+;pe-;[~p1&~c0e];c11-
    []pt&c1e->c1t+;pe-;[~pt&~c0e];c1t-
    ]
 ]
\end{hse}

%%%%%%%%%%%%%%%%%%%%%%%%%%%%%%%%%%%%%%%%
\subsubsection*{PRS}

%%%%%%%%%%%%%%%%%%%%%%%%%%%%%%%%%%%%%%%%%%%%%%%%%%%%%%%%%%%%
\subsection{READ\_HEAD}

READ\_HEAD reads the head word and signals FWD\_BODY which way 
to forward the body packet

%%%%%%%%%%%%%%%%%%%%%%%%%%%%%%%%%%%%%%%%
\subsubsection*{CHP}

%%%%%%%%%%%%%%%%%%%%%%%%%%%%%%%%%%%%%%%%
\subsubsection*{HSE}

% [[si];xe+;[xi];u+;xe-;[~xi];so+;[~si];u-;so-]
\begin{hse}
*[[si];xe+;
    [ x0->u0+;xe-;[~x0];s0+;[~si];u0-;s0-
    []x1->u1+;xe-;[~x1];s1+;[~si];u1-;s1-
    ]
 ]
\end{hse}

%%%%%%%%%%%%%%%%%%%%%%%%%%%%%%%%%%%%%%%%
\subsubsection*{PRS}

\begin{prs2}
si & ~u0 & ~u1 -> xe+
u0 | u1 -> xe-
\end{prs2}

\begin{prs2}
x0 -> u0+
~si -> u0-

x1 -> u1+
~si -> u1-

u0 & ~x0 -> s0+
~u0 & ~si -> s0-

u1 & ~x1 -> s1+
~u1 & ~si -> s1-
\end{prs2}

%%%%%%%%%%%%%%%%%%%%%%%%%%%%%%%%%%%%%%%%%%%%%%%%%%%%%%%%%%%%
\subsection{FWD\_BODY}

FWD\_BODY forwards words to the children based on command from DEC

%%%%%%%%%%%%%%%%%%%%%%%%%%%%%%%%%%%%%%%%
\subsubsection*{CHP}

%%%%%%%%%%%%%%%%%%%%%%%%%%%%%%%%%%%%%%%%
\subsubsection*{HSE}

% [so+;[si];pe+;[pi&ci];co+;pe-;[~pe&~ci];co-]
\begin{hse}
*[so+;[s0|s1];pe+;
    [ p0&s0&c0e->c00+;pe-;[~p0&~c0e];c00-
    []p1&s0&c0e->c01+;pe-;[~p1&~c0e];c01-
    []pt&s0&c0e->c0t+;pe-;[~pt&~c0e];so-;[~s0];c0t-
    []p0&s1&c1e->c10+;pe-;[~p0&~c1e];c10-
    []p1&s1&c1e->c11+;pe-;[~p1&~c1e];c11-
    []pt&s1&c1e->c1t+;pe-;[~pt&~c1e];so-;[~s1];c1t-
    ]
 ]
\end{hse}

%%%%%%%%%%%%%%%%%%%%%%%%%%%%%%%%%%%%%%%%
\subsubsection*{PRS}

\begin{prs2}
s0 | s1 & ~q -> pe+
q -> pe-

~q -> so+
~pt & (~c0e & c0t | ~c1e & c1t)-> so-

c00 | c01 | c0t | c10 | c11 | c1t -> q+
~c00 & ~c01 & ~c0t & ~c10 & ~c11 & ~c1t -> q-
\end{prs2}

\begin{prs2}
p0 & s0 & c0e -> c00+
~p0 & ~c0e -> c00-

p1 & s0 & c0e -> c01+
~p1 & ~c0e -> c01-

pt & s0 & c0e -> c0t+
~s0 -> c0t-

p0 & s1 & c1e -> c10+
~p0 & ~c0e -> c10-

p1 & s1 & c1e -> c11+
~p1 & ~c0e -> c11-

pt & s1 & c1e -> c1t+
~s1 -> c1t-
\end{prs2}

%%%%%%%%%%%%%%%%%%%%%%%%%%%%%%%%%%%%%%%%%%%%%%%%%%%%%%%%%%%%%%%%%%%%%%%%%%%%%%%
\end{document}
