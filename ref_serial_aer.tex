\documentclass{article}
\usepackage{mystyle}
\usepackage[ampersand]{easylist}

\begin{document}
\title{Serialized Tree AER Design Space}
\author{Sam Fok}
\maketitle

This document explores the design space of the serialized-tree AER system.
The system uses an address-event representation and transmitters/receivers with tree architectures.
The accounting is done assuming we're using an array of 4096 neurons.

\noindent Design parametrizations:

\begin{easylist}
\ListProperties(Progressive=1em, Hide=100, Style*=- )
    & N for 1-of-N encoding
\end{easylist}

\noindent We'll want to make sure that the designs can scale to 1-of-4 encoding.

\noindent \textbf{Transmitter (AEXT) design space:}
\begin{easylist}
    & active sender, passive receiver (ASPR)
    && monolithic NODE
    && decomposed into PFWD and MERGE
    &&& PFWD
	&&&& unpipelined (Section~\ref{sec:AEXT_ASPR_PM_PFWD_u})
	&&&& \textbf{pipelined} (Section~\ref{sec:AEXT_ASPR_PM_PFWD_u})
	&&&&& \textbf{hq} (Section~\ref{sec:AEXT_ASPR_PM_PFWD_p_hq})
	&&&&& \textbf{hu} (Section~\ref{sec:AEXT_ASPR_PM_PFWD_p_hu})
    &&&& decomposed into PREPEND, FWD, and SIMPLE\_MERGE
    &&&&& PREPEND
    &&&&& FWD
    &&&&& MERGE
    &&& MERGE
    &&&& unpipelined
    &&&& pipelined
	&&&&& \textbf{a\_a} (Section~\ref{sec:AEXT_ASPR_PM_MERGE_p_a_a})
	&&&&& \textbf{ah} (Section~\ref{sec:AEXT_ASPR_PM_MERGE_p_ah})
    & passive sender, active receiver (ASPR)
\end{easylist}

\noindent \textbf{Receiver (AERV) design space:}
\begin{easylist}
    & active sender, passive receiver (ASPR)
    && monolithic NODE
    && decomposed into BCAST and FILTER
    &&& BCAST
    &&&& unpipelined
    &&&& pipelined
    &&&&& \textbf{strict}
    &&&&& parallelized
    &&&&&& \textbf{cpcp} 
    &&&&&& cppc
    &&& FILTER 
    &&&& monolith
    &&&&& unpipelined
    &&&&& pipelined
    &&&& decomposed into BCAST, DECIDE, S\_FILTER
    &&&&& BCAST
    &&&&&& \textbf{unpipelined}
    &&&&& DECIDE
    &&&&&&  unpipelined?
    &&&&& S\_FILTER
    &&&&&& unpipelined
    & passive sender, active receiver (PSAR)
    && monolithic NODE
    && decomposed into ROUTE, READ\_HEAD, FWD\_BODY (Section~\ref{sec:AERV_PSAR_RHB})
    &&& ROUTE
    &&&& \textbf{unpipelined} (Section~\ref{sec:AERV_PSAR_RHB_ROUTE_u})
    &&& \textbf{READ\_HEAD} (Section~\ref{sec:AERV_PSAR_RHB_READ_HEAD})
    &&& FWD\_BODY
    &&&& \textbf{unpipelined} (Section~\ref{sec:AERV_PSAR_RHB_FWD_BODY_u})
    &&&& \textbf{pipelined} (Section~\ref{sec:AERV_PSAR_RHB_FWD_BODY_p})
    && decomposed into ROUTE, PULL\_CTRL, PULL (Section~\ref{sec:AERV_PSAR_RCP})
    &&& \textbf{ROUTE} (Section~\ref{sec:AERV_PSAR_RCP_ROUTE})
    &&& \textbf{PULL\_CTRL} (Section~\ref{sec:AERV_PSAR_RCP_PULL_CTRL})
    &&& PULL
    &&&& \textbf{unpipelined} (Section~\ref{sec:AERV_PSAR_RCP_PULL_u})
\end{easylist}
                
We also need the following interfaces:

\begin{easylist}
    & serial to parallel converter
    & parallel to serial converter
    & leaf interface
\end{easylist}

\noindent\makebox[\linewidth]{\rule{\textwidth}{1pt}}  

%%%%%%%%%%%%%%%%%%%%%%%%%%%%%%%%%%%%%%%%%%%%%%%%%%%%%%%%%%%%%%%%%%%%%%%%%%%%%%%
\section{AEXT ASPR NODE}

\begin{csp}
NODE\equiv
*[[h->
    [#{C0}->s:=0,P!(0);
    \|#{C1}->s:=1,P!(1)];
    h:=false
  []~h->
    [s=0->C0?x;P!x
    []s=1->C1?x;P!x
    ];h:=x.tail
 ]]
\end{csp}

%%%%%%%%%%%%%%%%%%%%%%%%%%%%%%%%%%%%%%%%%%%%%%%%%%%%%%%%%%%%%%%%%%%%%%%%%%%%%%%
\section{AEXT ASPR PFWD/MERGE (PM)}

PFWD prepends a word to the packet indicating which branch the packet is coming from.
MERGE arbitrates between incoming packet streams and outputs them one at a time.

\begin{csp}
PFWD\equiv
  h:=true;
  *[[h&#{X}->Y!(\textrm{header});h-
    []~h&#{X}->Y!(X?)\*[X=t->h+];
    ]
   ]
\end{csp}

\noindent Radix 2 accounting (4095 nodes / 4096 neurons):

\begin{center}
    \begin{tabular}{|r|l|l|l|}
    \hline
    component & transistors/component & components/node & transistors/node \\ \hline
    PFWD & 53 & 2 & 106 \\ \hline
    MERGE & 90 & 1 & 90 \\ \hline
    \multicolumn{3}{|r|}{total transistors/node} & 196 \\ \hline
    \end{tabular}
\end{center}

196 transistors/node * 4095 nodes / 4096 neurons = \textbf{196.0 transistors/neuron}

The leaf node PFWDs only need to communicate the tail bit and their prepend bit and can leave off the other bit.
This saves 2 production rules, or 14 transistors per node. With 2048 leaf nodes, this saves us 28672 transistors.

(196 transistors/node * 4095 nodes - 28672) / 4096 neurons = \textbf{189.0 transistors/neuron}

\noindent Radix 4 transistor accounting (1365 nodes / 4096 neurons): 

\begin{center}
    \begin{tabular}{|r|l|l|l|}
    \hline
    component & transistors/component & components/node & transistors/node \\ \hline
    PFWD & 73 & 4 & 292 \\ \hline
    MERGE & 288 & 1 & 288 \\ \hline
    \multicolumn{3}{|r|}{total transistors/node} & 580 \\ \hline
    \end{tabular}
\end{center}

580 transistors/node * 1365 nodes / 4096 neurons = \textbf{193.3 transistors/neuron}

The leaf node PFWDs only need to communicate the tail bit and their prepend bit and can leave off the other bit.
This saves 4 production rules, or 32 transistors per node. With 1024 leaf nodes, this saves us 32768 transistors.

(698 transistors/node * 4095 nodes - 32768) / 4096 neurons = \textbf{185.3 transistors/neuron}

%%%%%%%%%%%%%%%%%%%%%%%%%%%%%%%%%%%%%%%%%%%%%%%%%%%%%%%%%%%%%%%%%%%%%%%%%%%%%%%
\section{AEXT ASPR PM PFWD unpipelined \label{sec:AEXT_ASPR_PM_PFWD_u}}

%%%%%%%%%%%%%%%%%%%%%%%%%%%%%%%%%%%%%%%%
\subsubsection*{CHP}

%%%%%%%%%%%%%%%%%%%%%%%%%%%%%%%%%%%%%%%%
\subsubsection*{HSE}

\begin{hse}
*[h&(x0|x1|xt)->q+;yp+;[yi];h-;yp-;[~yi];q-
  []~h&~q&x0->y0+;[yi];xo+;[~x0];y0-;[~yi];xo-
  []~h&~q&x1->y1+;[yi];xo+;[~x1];y1-;[~yi];xo-
  []~h&~q&xt->yt+;[yi];xo+;[~xt];h+;yt-;[~yi];xo-]
\end{hse}

%%%%%%%%%%%%%%%%%%%%%%%%%%%%%%%%%%%%%%%%
\subsubsection*{PRS}

\begin{prs2}
~xt & yt -> h+
q & yi -> h-

h & (x0 | x1 | xt) -> q+
~h & ~yi -> q-
\end{prs2}

\begin{prs2}
~q & yi -> xo+
q | ~yi -> xo-
\end{prs2}

\begin{prs2}
h & q -> yp+
~h & q & yi -> yp-
\end{prs2}

\begin{prs2}
~h & ~q & x0 -> y0+
~q & ~x0 -> y0-

~h & ~q & x1 -> y1+
~q & ~x1 -> y1-

~h & ~q & xt -> yt+
h -> yt-
\end{prs2}

\noindent Radix 2 transistor accounting: 

\begin{center}
    \begin{tabular}{|r|l|l|}
    \hline
    rule & transistor count & comments \\ \hline
    $h$ & 8 & \\ \hline
    $q$ & 10 & \\ \hline
    $x_o$ & 4 & \\ \hline
    $yp$ & 5 & OR'ed with a $y$ rule below which also provides the staticizer \\ \hline
    $y[0,1]$ & 18 & \\ \hline
    $yt$ & 8 & \\ \hline
    \hline total & 53 & \\ \hline
    \end{tabular}
\end{center}

\noindent Radix 4 transistor accounting: 

\begin{center}
    \begin{tabular}{|r|l|l|}
    \hline
    rule & transistor count & comments \\ \hline
    $h$ & 8 & \\ \hline
    $q$ & 12 & \\ \hline
    $x_o$ & 4 & \\ \hline
    $yp$ & 5 & OR'ed with a $y$ rule below which also provides the staticizer \\ \hline
    $y[0,1,2,3]$ & 36 & \\ \hline
    $yt$ & 8 & \\ \hline
    \hline total & 73 & \\ \hline
    \end{tabular}
\end{center}

%%%%%%%%%%%%%%%%%%%%%%%%%%%%%%%%%%%%%%%%%%%%%%%%%%%%%%%%%%%%%%%%%%%%%%%%%%%%%%%
\section{AEXT ASPR PM PFWD pipelined hq \label{sec:AEXT_ASPR_PM_PFWD_p_hq}}

This version has fewer state variables than AEXT PFWD hu, but the pull-up and pull-down chains are too long.

%%%%%%%%%%%%%%%%%%%%%%%%%%%%%%%%%%%%%%%%
\subsubsection*{CHP}

\begin{csp}
PFWD\equiv
  h:=true;
  *[[h&#{X}->Y!(\textrm{header});h-
    []~h&#{X}->Y!(X?)\*[X=t->h+];
    ]
   ]
\end{csp}

%%%%%%%%%%%%%%%%%%%%%%%%%%%%%%%%%%%%%%%%
\subsubsection*{HSE}

\begin{hse}
*[h&(x0|x1|xt)->yp+;[yi];q+;yp-;[~yi];h-;q-
  []~h&x0->y0+;xo+;[yi];y0-;[~x0];xo-;[~yi]
  []~h&x1->y1+;xo+;[yi];y1-;[~x1];xo-;[~yi]
  []~h&xt->yt+;xo+;[yi];h+;yt-;[~xt];xo-;[~yi]]
\end{hse}

%%%%%%%%%%%%%%%%%%%%%%%%%%%%%%%%%%%%%%%%
\subsubsection*{PRS}

\begin{prs2}
h & yi & yp -> q+
~h -> q-

yt & xo & yi -> h+
q & ~yi -> h-
\end{prs2}

\begin{prs2}
~h & (y0 | y1 | yt) -> xo+
~x0 & ~x1 & ~xt & ~y0 & ~y1 & ~yt -> xo-
\end{prs2}

\begin{prs2}
h & ~q & ~yi & ~xo & (x0 | xi | xt) -> yp+
q -> yp-
\end{prs2}

\begin{prs2}
~h & ~q & ~yi & x0 & ~xo -> y0+
yi & xo -> y0-

~h & ~q & ~yi & x1 & ~xo -> y1+
yi & xo -> y1-
\end{prs2}

\begin{prs2}
~h & ~q & ~yi & xt & ~xo -> yt+
h & xo -> yt-
\end{prs2}

\noindent Radix 2 transistor accounting: 

\begin{center}
    \begin{tabular}{|r|l|l|}
    \hline
    rule & transistor count & comments \\ \hline
    $q$ & 8 & \\ \hline
    $h$ & 9 & \\ \hline
    $xo$ & 14 & \\ \hline
    $yp$ & 8 & OR'ed with a $y$ rule below which also provides the staticizer \\ \hline
    $y[0,1]$ & 22 & \\ \hline
    $yt$ & 11 & \\ \hline
    \hline total & 72 & \\ \hline
    \end{tabular}
\end{center}

\noindent Radix 4 transistor accounting: 

\begin{center}
    \begin{tabular}{|r|l|l|}
    \hline
    rule & transistor count & comments \\ \hline
    $q$ & 8 & \\ \hline
    $h$ & 9 & \\ \hline
    $xo$ & 20 & \\ \hline
    $yp$ & 10 & OR'ed with a $y$ rule below which also provides the staticizer \\ \hline
    $y[0,1,2,3]$ & 44 & \\ \hline
    $yt$ & 11 & \\ \hline
    \hline total & 102 & \\ \hline
    \end{tabular}
\end{center}

%%%%%%%%%%%%%%%%%%%%%%%%%%%%%%%%%%%%%%%%%%%%%%%%%%%%%%%%%%%%%%%%%%%%%%%%%%%%%%%
\section{AEXT ASPR PM PFWD hu \label{sec:AEXT_ASPR_PM_PFWD_p_hu}}

This version has more state variables than AEXT PFWD hq, but has reasonable pull-up and pull-down chains.

%%%%%%%%%%%%%%%%%%%%%%%%%%%%%%%%%%%%%%%%
\subsubsection*{CHP}

\begin{csp}
PFWD\equiv
  *[[h&#{X}->Y!(\textrm{header});h-
    []~h&#{X}->X?u\*Y!u,[u=t->h+]
    ]
   ]
\end{csp}

%%%%%%%%%%%%%%%%%%%%%%%%%%%%%%%%%%%%%%%%
\subsubsection*{HSE}

\begin{hse}
*[[h&(x0|x1|xt)->yp+;[yi];h-;yp-;[~yi]
  []~h&x0->u0+;(xo+;[~x0]),(y0+;[yi]);u0-;(y0-;[~yi]),xo-
  []~h&x1->u1+;(xo+;[~x1]),(y1+;[yi]);u1-;(y1-;[~yi]),xo-
  []~h&xt->ut+;(xo+;[~xt]),(yt+;h+;[yi]);ut-;(yt-;[~yi]),xo-
  ]
 ]
\end{hse}

\begin{hse}
*[[h&(x0|x1|xt)->yp+;[yi];h-;yp-;[~yi]
  []~h&x0->u0+;[~x0&yi];u0-;[~yi]
  []~h&x1->u1+;[~x1&yi];u1-;[~yi]
  []~h&xt->ut+;[~xt&h&yi];ut-;[~yi]
  ]
 ]

*[u0->xo+,y0+;[~u0];y0-,xo-
  []u1->xo+,y1+;[~u1];y1-,xo-
  []ut->xo+,(yt+;h+);[ut-];yt-,xo-
  ]
\end{hse}

%%%%%%%%%%%%%%%%%%%%%%%%%%%%%%%%%%%%%%%%
\subsubsection*{PRS}

\begin{prs2}
yt -> h+
yp & yi -> h-
\end{prs2}

\begin{prs2}
h & (x0 | x1 | xt) & ~yi & ~yt -> yp+
~h & yi & ~un -> yp-
\end{prs2}

\begin{prs2}
u0 | u1 | ut -> xo+
~u0 & ~u1 & ~ut -> xo-
\end{prs2}

\begin{prs2}
~h & x0 & ~yi -> u0+
~x0 & yi -> u0-

~h & x1 & ~yi -> u1+
~x1 & yi -> u1-
\end{prs2}

\begin{prs2}
~h & xt & ~yi -> ut+
h & ~xt & yi -> ut-
\end{prs2}

\begin{prs2}
u0 -> y0+
~h & ~u0 -> y0-

u1 -> y1+
~h & ~u1 -> y1-
\end{prs2}

\begin{prs2}
ut -> yt+
~ut -> yt-
\end{prs2}

\noindent Radix 2 transistor accounting: 

\begin{center}
    \begin{tabular}{|r|l|l|}
    \hline
    rule & transistor count & comments \\ \hline
    $h$ & 7 & \\ \hline
    $yp$ & 9 & OR'ed with a $y$ rule below which also provides the staticizer \\ \hline
    $xo$ & 6 & \\ \hline
    $u[0,1]$ & 18 & \\ \hline
    $ut$ & 10 & \\ \hline
    $y[0,1]$ & 14 & \\ \hline
    $yt$ & 6 & \\ \hline
    \hline total & 70 & \\ \hline
    \end{tabular}
\end{center}

\noindent Radix 4 transistor accounting: 

\begin{center}
    \begin{tabular}{|r|l|l|}
    \hline
    rule & transistor count & comments \\ \hline
    $h$ & 7 & \\ \hline
    $yp$ & 11 & OR'ed with a $y$ rule below which also provides the staticizer \\ \hline
    $xo$ & 10 & \\ \hline
    $u[0,1,2,3]$ & 36 & \\ \hline
    $ut$ & 10 & \\ \hline
    $y[0,1,2,3]$ & 28 & \\ \hline
    $yt$ & 6 & \\ \hline
    \hline total & 108 & \\ \hline
    \end{tabular}
\end{center}

%%%%%%%%%%%%%%%%%%%%%%%%%%%%%%%%%%%%%%%%%%%%%%%%%%%%%%%%%%%%%%%%%%%%%%%%%%%%%%%
\section{AEXT ASPR PM MERGE unpipelined \label{sec:AEXT_ASPR_PM_MERGE_u}}

%%%%%%%%%%%%%%%%%%%%%%%%%%%%%%%%%%%%%%%%
\subsubsection*{CHP}

%%%%%%%%%%%%%%%%%%%%%%%%%%%%%%%%%%%%%%%%
\subsubsection*{HSE}

\begin{hse}
*[[~a0&~a1&(c00|c01|c0t)->a0+
  \|~a0&~a1&(c10|c11|c1t)->a1+]]

*[[ a0&c00->p0+;[pi];c0o+;[~c00];p0-;[~pi];c0o-
  []a0&c01->p1+;[pi];c0o+;[~c01];p1-;[~pi];c0o-
  []a0&c0t->pt+;[pi];c0o+;[~c0t];a0-;pt-;[~pi];c0o-
  []a1&c10->p0+;[pi];c1o+;[~c10];p0-;[~pi];c1o-
  []a1&c11->p1+;[pi];c1o+;[~c11];p1-;[~pi];c1o-
  []a1&c1t->pt+;[pi];c1o+;[~c1t];a1-;pt-;[~pi];c1o-
  ]]
\end{hse}

%%%%%%%%%%%%%%%%%%%%%%%%%%%%%%%%%%%%%%%%
\subsubsection*{PRS}

\begin{prs2}
(c00 | c01| c0t) & ~a0 -> a0i+
a0 -> a0i-

(c10 | c11| c1t) & ~a1-> a1i+
a1 -> a1i-

a0o & ~a1 & ~pi -> a0+
~a0o & pt & ~c0t -> a0-

a1o & ~a0 & ~pi -> a1+
~a1o & pt & ~c1t -> a1-
\end{prs2}

\begin{prs2}
(a0 & c00 | a1 & c10) -> p0+
(a0 & ~c00 | a1 & ~c10) -> p0-

(a0 & c00 | a1 & c10) -> p0+
(a0 & ~c00 | a1 & ~c10) -> p0-

(a0 & c0t | a1 & c1t) -> pt+
~a0 & ~a1 -> pt-
\end{prs2}

\begin{prs2}
a0 & pi -> c0o+
~a0 | ~pi -> c0o-

a1 & pi -> c1o+
~a1 | ~pi -> c1o-
\end{prs2}

\noindent Radix 2 transistor accounting: 

\begin{center}
    \begin{tabular}{|r|l|l|}
    \hline
    rule & transistor count & comments \\ \hline
    $a[0,1]i$ & 16 & can be combinational with radix 2 \\ \hline
    $a[0,1]o$ & 12 & 2-way arbiter \\ \hline
    $a[0,1]$ & 20 & \\ \hline
    $p[0,1]$ & 24 & \\ \hline
    $pt$ & 10 & \\ \hline
    $c[0,1]o$ & 8 & \\ \hline
    \hline total & 90 & \\ \hline
    \end{tabular}
\end{center}

\noindent Radix 4 transistor accounting: 

\begin{center}
    \begin{tabular}{|r|l|l|}
    \hline
    rule & transistor count & comments \\ \hline
    $a[0,1,2,3]i$ & 44 & \\ \hline
    $a[0,1,2,3]o$ & 92 & 4-way unpipelined arbiter \\ \hline
    $a[0,1,2,3]$ & 40 & \\ \hline
    $p[0,1,2,3]$ & 80 & \\ \hline
    $pt$ & 16 & \\ \hline
    $c[0,1,2,3]o$ & 16 & \\ \hline
    \hline total & 288 & \\ \hline
    \end{tabular}
\end{center}

%%%%%%%%%%%%%%%%%%%%%%%%%%%%%%%%%%%%%%%%%%%%%%%%%%%%%%%%%%%%%%%%%%%%%%%%%%%%%%%
\section{AEXT ASPR PM MERGE pipelined a\_a \label{sec:AEXT_ASPR_PM_MERGE_p_a_a}}

I don't like this version because the pullup chains for the state variables
are too long and won't scale to higher radix encoding.

%%%%%%%%%%%%%%%%%%%%%%%%%%%%%%%%%%%%%%%%
\subsubsection*{CHP}

\begin{csp}
MERGE\equiv
  *[[h->[#{C0}->a:=0\|#{C1}->a:=1];h-
    []~h&a=0->P!(C0?)
    []~h&a=0->P!(C1?)
    ]]
\end{csp}

%%%%%%%%%%%%%%%%%%%%%%%%%%%%%%%%%%%%%%%%
\subsubsection*{HSE}

\begin{hse}
*[[~a0&~a1&(c00|c01|c0t)->a0+
  \|~a0&~a1&(c10|c11|c1t)->a1+]]

*[[ a0&c00->p0+;c0o+;[pi&~c00];p0-;c0o-;[~pi]
  []a0&c01->p1+;c0o+;[pi&~c01];p1-;c0o-;[~pi]
  []a0&c0t->pt+;c0o+;[pi&~c0t];a0-;pt-;c0o-;[~pi]
  []a1&c10->p0+;c1o+;[pi&~c10];p0-;c1o-;[~pi]
  []a1&c11->p1+;c1o+;[pi&~c11];p1-;c1o-;[~pi]
  []a1&c1t->pt+;c1o+;[pi&~c1t];a1-;pt-;c1o-;[~pi]
  ]]
\end{hse}

%%%%%%%%%%%%%%%%%%%%%%%%%%%%%%%%%%%%%%%%
\subsubsection*{PRS}

\begin{prs2}
~a0 & (c00 | c01| c0t) & ~c0o -> a0i+
a0 -> a0i-

~a1 & (c10 | c11| c1t) & ~c1o -> a1i+
a1 -> a1i-

a0o & ~a1 & ~c1o & ~pi -> a0+
~a0o & pi & pt & ~c0t -> a0-

a1o & ~a0 & ~c0o & ~pi -> a1+
~a1o & pi & pt & ~c1t -> a1-
\end{prs2}

\begin{prs2}
~pi & (a0 & c00 | a1 & c10) -> p0+
pi & (~_a0 & ~c00 | ~_a1 & ~c10) -> p0-

~pi & (a0 & c01 | a1 & c11) -> p1+
pi & (~_a0 & ~c01 | ~_a1 & ~c11) -> p1-

~pi & (a0 & c0t | a1 & c1t) -> pt+
_a0 & _a1 -> pt-
\end{prs2}

\begin{prs2}
a0 & (p0 | p1 | pt) -> c0o+
~p0 & ~p1 & ~pt -> c0o-

a1 & (p0 | p1 | pt) -> c1o+
~p0 & ~p1 & ~pt -> c1o-
\end{prs2}

\noindent Radix 2 transistor accounting: 

\begin{center}
    \begin{tabular}{|r|l|l|}
    \hline
    rule & transistor count & comments \\ \hline
    $a[0,1]i$ & 20 & \\ \hline
    $a[0,1]o$ & 12 & 2-way arbiter \\ \hline
    $a[0,1]$ & 24 & \\ \hline
    $p[0,1]$ & 28 & \\ \hline
    $pt$ & 11 & \\ \hline
    $c[0,1]o$ & 22 & \\ \hline
    \hline total & 117 & \\ \hline
    \end{tabular}
\end{center}

\noindent Radix 4 transistor accounting: 

\begin{center}
    \begin{tabular}{|r|l|l|}
    \hline
    rule & transistor count & comments \\ \hline
    $a[0,1,2,3]i$ & 48 & \\ \hline
    $a[0,1,2,3]o$ & 92 & 4-way unpipelined arbiter \\ \hline
    $a[0,1,2,3]$ & 48 & \\ \hline
    $p[0,1,2,3]$ & 88 & \\ \hline
    $pt$ & 17 & \\ \hline
    $c[0,1,2,3]o$ & 60 & \\ \hline
    \hline total & 353 & \\ \hline
    \end{tabular}
\end{center}

%%%%%%%%%%%%%%%%%%%%%%%%%%%%%%%%%%%%%%%%%%%%%%%%%%%%%%%%%%%%%%%%%%%%%%%%%%%%%%%
\section{AEXT ASPR PM MERGE pipelined ah \label{sec:AEXT_ASPR_PM_MERGE_p_ah}}

This one has acceptable pullup/pulldown chains, but I'm worried about making it CMOS implementable

%%%%%%%%%%%%%%%%%%%%%%%%%%%%%%%%%%%%%%%%
\subsubsection*{HSE}

\begin{hse}
*[[~a0&(c00|c01|c0t)->a0+;h-;[~a0];h+
  \|~a1&(c10|c11|c1t)->a1+;h-;[~a1];h+]]

*[[ a0&c00->p0+;c0o+;[pi&~c00];p0-;c0o-;[~pi]
  []a0&c01->p1+;c0o+;[pi&~c01];p1-;c0o-;[~pi]
  []a0&c0t->pt+;c0o+;[pi&~c0t];a0-;pt-;c0o-;[~pi]
  []a1&c10->p0+;c1o+;[pi&~c10];p0-;c1o-;[~pi]
  []a1&c11->p1+;c1o+;[pi&~c11];p1-;c1o-;[~pi]
  []a1&c1t->pt+;c1o+;[pi&~c1t];a1-;pt-;c1o-;[~pi]
  ]]
\end{hse}

%%%%%%%%%%%%%%%%%%%%%%%%%%%%%%%%%%%%%%%%
\subsubsection*{PRS}

\begin{prs2}
(c00 | c01| c0t) & ~a0 -> a0i+
~h & a0 -> a0i-

(c10 | c11| c1t) & ~a1-> a1i+
~h & a1 -> a1i-

h & a0o & ~pi -> a0+
pt & pi & c0o & ~c0t & ~a0o -> a0-

h & a1o & ~pi -> a1+
pt & pi & c1o & ~c1t & ~a1o -> a1-
\end{prs2}

\begin{prs2}
~a0 & ~a1 -> h+
a0 | a1 & ~c0o & ~c1o-> h-
\end{prs2}

\begin{prs2}
~pi & (a0 & c00 | a1 & c10) & ~h -> p0+
pi & (a0 & ~c00 | a1 & ~c10) -> p0-

~pi & (a0 & c01 | a1 & c11) & ~h -> p1+
pi & (a0 & ~c01 | a1 & ~c11) -> p1-

~pi & (a0 & c0t | a1 & c1t) & ~h -> pt+
~a0 & ~a1 -> pt-
\end{prs2}

\begin{prs2}
a0 & (p0 | p1 | pt) -> c0o+
~p0 & ~p1 & ~pt -> c0o-

a1 & (p0 | p1 | pt) -> c1o+
~p0 & ~p1 & ~pt -> c1o-
\end{prs2}

\noindent Radix 2 transistor accounting: 

\begin{center}
    \begin{tabular}{|r|l|l|}
    \hline
    rule & transistor count & comments \\ \hline
    $a[0,1]i$ & 20 & \\ \hline
    $a[0,1]o$ & 12 & 2-way arbiter \\ \hline
    $a[0,1]$ & 24 & \\ \hline
    $h$ & 10 & \\ \hline
    $p[0,1]$ & 30 & \\ \hline
    $pt$ & 12 & \\ \hline
    $c[0,1]o$ & 22 & \\ \hline
    \hline total & 130 & \\ \hline
    \end{tabular}
\end{center}

\noindent Radix 4 transistor accounting: 

\begin{center}
    \begin{tabular}{|r|l|l|}
    \hline
    rule & transistor count & comments \\ \hline
    $a[0,1,2,3]i$ & 48 & \\ \hline
    $a[0,1,2,3]o$ & 92 & 4-way unpipelined arbiter \\ \hline
    $a[0,1,2,3]$ & 48 & \\ \hline
    $h$ & 14 & \\ \hline
    $p[0,1,2,3]$ & 52 & \\ \hline
    $pt$ & 16 & \\ \hline
    $c[0,1,2,3]o$ & 100 & \\ \hline
    \hline total & 370 & \\ \hline
    \end{tabular}
\end{center}

%%%%%%%%%%%%%%%%%%%%%%%%%%%%%%%%%%%%%%%%%%%%%%%%%%%%%%%%%%%%%%%%%%%%%%%%%%%%%%%
\section{AEXT ASPR PFWD PREPEND/FWD/SIMPLE\_MERGE}

PREPEND and FWD are

%%%%%%%%%%%%%%%%%%%%%%%%%%%%%%%%%%%%%%%%%%%%%%%%%%%%%%%%%%%%%%%%%%%%%%%%%%%%%%%
\section{AEXT ASPR PFWD PREPEND}

%%%%%%%%%%%%%%%%%%%%%%%%%%%%%%%%%%%%%%%%
\subsubsection*{CHP}

%%%%%%%%%%%%%%%%%%%%%%%%%%%%%%%%%%%%%%%%
\subsubsection*{HSE}
\begin{hse}
*[[si];yp+;[yi];yp-;[~yi];so+;[~si];so-]
\end{hse}

%%%%%%%%%%%%%%%%%%%%%%%%%%%%%%%%%%%%%%%%
\subsubsection*{PRS}

%%%%%%%%%%%%%%%%%%%%%%%%%%%%%%%%%%%%%%%%%%%%%%%%%%%%%%%%%%%%%%%%%%%%%%%%%%%%%%%
\section{AEXT PFWD FWD}

%%%%%%%%%%%%%%%%%%%%%%%%%%%%%%%%%%%%%%%%
\subsubsection*{CHP}

%%%%%%%%%%%%%%%%%%%%%%%%%%%%%%%%%%%%%%%%
\subsubsection*{HSE}

\begin{hse}
*[[x0|x1|xt];so+;[si];so-;[~si];
  [x0->y0+;[yi];y0-;[~yi]
  []x1->y1+;[yi];y1-;[~yi]
  []xt->yt+;[yi];yt-;[~yi]
 ]]
\end{hse}

%%%%%%%%%%%%%%%%%%%%%%%%%%%%%%%%%%%%%%%%
\subsubsection*{PRS}

%%%%%%%%%%%%%%%%%%%%%%%%%%%%%%%%%%%%%%%%%%%%%%%%%%%%%%%%%%%%%%%%%%%%%%%%%%%%%%%
\section{AEXT ASPR PFWD SIMPLE\_MERGE}

Assumes that its inputs are mutually exclusive

%%%%%%%%%%%%%%%%%%%%%%%%%%%%%%%%%%%%%%%%
\subsubsection*{CHP}

%%%%%%%%%%%%%%%%%%%%%%%%%%%%%%%%%%%%%%%%
\subsubsection*{HSE}

\begin{hse}
*[[c00->p0+;[pi];c0o+;[~c00];p0-;[~pi];c0o-
  []c01->p1+;[pi];c0o+;[~c01];p1-;[~pi];c0o-
  []c0t->pt+;[pi];c0o+;[~c0t];pt-;[~pi];c0o-
  []c10->p0+;[pi];c1o+;[~c00];p0-;[~pi];c1o-
  []c11->p1+;[pi];c1o+;[~c01];p1-;[~pi];c1o-
  []c1t->pt+;[pi];c1o+;[~c0t];pt-;[~pi];c1o-
 ]]
\end{hse}

%%%%%%%%%%%%%%%%%%%%%%%%%%%%%%%%%%%%%%%%
\subsubsection*{PRS}

\begin{prs2}
c00 | c10 -> p0+
~c00 & ~c10 -> p0-

c01 | c11 -> p1+
~c01 & ~c11 -> p1-

c0t | c1t -> pt+
~c0t & ~c1t -> pt-
\end{prs2}

\begin{prs2}
pi & (c00 | c01 | c0t) -> c0o+
~pi -> c0o-

pi & (c10 | c11 | c1t) -> c1o+
~pi -> c1o-
\end{prs2}

%%%%%%%%%%%%%%%%%%%%%%%%%%%%%%%%%%%%%%%%%%%%%%%%%%%%%%%%%%%%%%%%%%%%%%%%%%%%%%%
\section{AEXT ASPR MERGE}

MERGE sequences between outputting two serialized packet streams.

\begin{csp}
MERGE\equiv
  *[[~l&~r->
      [#{L}->l:=true
      \|#{R}->r:=true
      ]
    []l|r->
      [l->O!(L?)
      []r->O!(R?)
    ]]
  ]
\end{csp}

\noindent\makebox[\linewidth]{\rule{\textwidth}{1pt}}  

%%%%%%%%%%%%%%%%%%%%%%%%%%%%%%%%%%%%%%%%%%%%%%%%%%%%%%%%%%%%%%%%%%%%%%%%%%%%%%%
\section{AERV ASPR BCAST pipelined}

%%%%%%%%%%%%%%%%%%%%%%%%%%%%%%%%%%%%%%%%
\subsubsection*{CHP}

%%%%%%%%%%%%%%%%%%%%%%%%%%%%%%%%%%%%%%%%
\subsubsection*{HSE}

strict cpcp

\begin{hse}
BCAST\equiv
*[[ p0->c00+,c10+;po+;[c0i&c1i&~p0];c00-,c10-;po-;[~c0i&~c1i];
  []p1->c01+,c11+;po+;[c0i&c1i&~p1];c01-,c11-;po-;[~c0i&~c1i];
  []pt->c0t+,c1t+;po+;[c0i&c1i&~pt];c0t-,c1t-;po-;[~c0i&~c1i];
 ]]
\end{hse}

\begin{hse}
BCAST\equiv
*[[ p0->q0+;c00+,c10+;po+;[c0i&c1i&~p0];q0-;c00-,c10-;po-;[~c0i&~c1i]
  []p1->q1+;c01+,c11+;po+;[c0i&c1i&~p1];q1-;c01-,c11-;po-;[~c0i&~c1i]
  []pt->qt+;c0t+,c1t+;po+;[c0i&c1i&~pt];qt-;c0t-,c1t-;po-;[~c0i&~c1i]
 ]]
\end{hse}

%%%%%%%%%%%%%%%%%%%%%%%%%%%%%%%%%%%%%%%%
\subsubsection*{PRS}

\begin{prs2}
q0 -> c00+
~q0 -> c00-

q1 -> c01+
~q1 -> c01-

qt -> c0t+
~qt -> c0t-

q0 -> c10+
~q0 -> c10-

q1 -> c11+
~q1 -> c11-

qt -> c1t+
~qt -> c1t-
\end{prs2}

\begin{prs2}
~c0i & ~c1i & p0 -> q0+
c0i & c1i & ~p0 -> q0-

~c0i & ~c1i & p1 -> q1+
c0i & c1i & ~p1 -> q1-

~c0i & ~c1i & pt -> qt+
c0i & c1i & ~pt -> qt-
\end{prs2}

%%%%%%%%%%%%%%%%%%%%%%%%%%%%%%%%%%%%%%%%
\subsubsection*{HSE}

output ordering cpcp
parallelized

\begin{hse}
BCAST\equiv
*[[ p0->([~c0i];c00+),([~c1i];c10+);po+;[~p0];([c0i];c00-),([c1i];c10-);po-;
  []p1->([~c0i];c01+),([~c1i];c11+);po+;[~p1];([c0i];c01-),([c1i];c11-);po-;
  []pt->([~c0i];c0t+),([~c1i];c1t+);po+;[~pt];([c0i];c0t-),([c1i];c1t-);po-;
 ]]
\end{hse}

%%%%%%%%%%%%%%%%%%%%%%%%%%%%%%%%%%%%%%%%
\subsubsection*{PRS}

\begin{prs2}
~c0i & p0 -> c00+
c0i & ~p0 -> c00-

~c0i & p1 -> c01+
c0i & ~p1 -> c01-

~c0i & pt -> c0t+
c0i & ~pt -> c0t-

~c1i & p0 -> c10+
c1i & ~p0 -> c10-

~c1i & p1 -> c11+
c1i & ~p1 -> c11-

~c1i & pt -> c1t+
c1i & ~pt -> c1t-
\end{prs2}

\begin{prs2}
VN(C) -> po+
~VN(C) -> po-
\end{prs2}

po is the output of a VN detector

%%%%%%%%%%%%%%%%%%%%%%%%%%%%%%%%%%%%%%%%
\subsubsection*{HSE}

swap ordering of p and c in reset
output ordering cppc
parallelized

\begin{hse}
BCAST\equiv
*[[ p0->([~c0i];c00+),([~c1i];c10+);(po+;[~p0];po-);([c0i];c00-),([c1i];c10-)
  []p1->([~c0i];c01+),([~c1i];c11+);(po+;[~p1];po-);([c0i];c01-),([c1i];c11-)
  []pt->([~c0i];c0t+),([~c1i];c1t+);(po+;[~pt];po-);([c0i];c0t-),([c1i];c1t-)
 ]]
\end{hse}

%%%%%%%%%%%%%%%%%%%%%%%%%%%%%%%%%%%%%%%%
\subsubsection*{PRS}

\begin{prs2}
~c0i & p0 -> c00+
c0i & ~p0 & ~po -> c00-

~c0i & p1 -> c01+
c0i & ~p1 & ~po -> c01-

~c0i & pt -> c0t+
c0i & ~pt & ~po -> c0t-

~c1i & p0 -> c10+
c1i & ~p0 & ~po -> c10-

~c1i & p1 -> c11+
c1i & ~p1 & ~po -> c11-

~c1i & pt -> c1t+
c1i & ~pt & ~po -> c1t-
\end{prs2}

\begin{prs2}
(c00 & c10 | c01 & c11 | c0t & c1t) & (p0 | p1 | pt) -> po+
~p0 & ~p1 & ~pt -> po-
\end{prs2}

instability on down phases of c because p input can rise at anytime
Could probably fix with state variables.

%%%%%%%%%%%%%%%%%%%%%%%%%%%%%%%%%%%%%%%%%%%%%%%%%%%%%%%%%%%%%%%%%%%%%%%%%%%%%%%
\section{AERV ASPR BCAST unpipelined}

%%%%%%%%%%%%%%%%%%%%%%%%%%%%%%%%%%%%%%%%
\subsubsection*{CHP}

%%%%%%%%%%%%%%%%%%%%%%%%%%%%%%%%%%%%%%%%
\subsubsection*{HSE}

\begin{hse}
BCAST\equiv
*[[ p0->c00+,c10+;[c0i&c1i];po+;[~p0];c00-,c10-;[~c0i&~c1i];po-
  []p1->c01+,c11+;[c0i&c1i];po+;[~p1];c01-,c11-;[~c0i&~c1i];po-
  []pt->c0t+,c1t+;[c0i&c1i];po+;[~pt];c0t-,c1t-;[~c0i&~c1i];po-
 ]]
\end{hse}

%%%%%%%%%%%%%%%%%%%%%%%%%%%%%%%%%%%%%%%%
\subsubsection*{PRS}

\begin{prs2}
p0 -> c00+
~p0 -> c00-

p1 -> c01+
~p1 -> c01-

pt -> c0t+
~pt -> c0t-

p0 -> c10+
~p0 -> c10-

p1 -> c11+
~p1 -> c11-

pt -> c1t+
~pt -> c1t-
\end{prs2}

\begin{prs2}
c0i & c1i -> po+
~c0i & ~c1i -> po-
\end{prs2}

%%%%%%%%%%%%%%%%%%%%%%%%%%%%%%%%%%%%%%%%%%%%%%%%%%%%%%%%%%%%%%%%%%%%%%%%%%%%%%%
\section{AERV PSAR}

This makes the circuitry much simpler

%%%%%%%%%%%%%%%%%%%%%%%%%%%%%%%%%%%%%%%%%%%%%%%%%%%%%%%%%%%%%%%%%%%%%%%%%%%%%%%
\section{AERV PSAR decomposed into ROUTE, READ\_HEAD, FWD\_BODY (RHB) \label{sec:AERV_PSAR_RHB}}

ROUTE sends a parent's signal to one of its children depending on which child requests. Assumes requests are mutually exclusive.

READ\_HEAD reads the head word and signals FWD\_BODY which way 
to forward the body packet

FWD\_BODY forwards words to the children based on command from DEC \\ 

\noindent Radix 2 accounting (4095 nodes / 4096 neurons):

\begin{center}
    \begin{tabular}{|r|l|l|l|}
    \hline
    component & transistors/component & components/node & transistors/node \\ \hline
    ROUTE & 39 & 1 & 39 \\ \hline
    READ\_HEAD & 29 & 1 & 29 \\ \hline
    FWD\_BODY & 51 & 1 & 51 \\ \hline
    \hline \multicolumn{3}{|r|}{total transistors/node} & 119 \\ \hline
    \end{tabular}
\end{center}

119 transistors/node * 4095 nodes / 4096 neurons = \textbf{119.0 transistors/neuron}

\noindent Radix 4 transistor accounting (1365 nodes / 4096 neurons): 

\begin{center}
    \begin{tabular}{|r|l|l|l|}
    \hline
    component & transistors/component & components/node & transistors/node \\ \hline
    ROUTE & 71 & 1 & 71 \\ \hline
    READ\_HEAD & 53 & 1 & 53 \\ \hline
    FWD\_BODY & 125 & 1 & 125 \\ \hline
    \hline \multicolumn{3}{|r|}{total transistors/node} & 249 \\ \hline
    \end{tabular}
\end{center}

249 transistors/node * 1365 nodes / 4096 neurons = \textbf{83.0 transistors/neuron}

However, we can still send a payload to the neurons with 1-of-2 data instead of 1-of-4 data at the leaf nodes.
There are 1024 leaf nodes. 
This will simplify the leaf node ROUTE and FWD\_BODY components because their children only need to see the 1-bit payload and tail.
Each leaf node ROUTE can lose 1 bit (i.e. 2 data lines or 2 asymmetric c-elements or 14 transistors).
Each leaf node FWD\_BODY can lose 1 bit (i.e. 2 data lines or 2 AND-gates or 8 transistors) for each of 4 children.
Therefore we can subtract 

1024*(14+8*4) = 47104 transistors.

\noindent Leaving out the high bit from the leaf nodes yields

(249 transistors/node * 1365 nodes - 47104 transistors) / 4096 neurons = \textbf{71.5 transistors/neuron}
%%%%%%%%%%%%%%%%%%%%%%%%%%%%%%%%%%%%%%%%%%%%%%%%%%%%%%%%%%%%%%%%%%%%%%%%%%%%%%%
\section{AERV PSAR RHB ROUTE unpipelined \label{sec:AERV_PSAR_RHB_ROUTE_u}}

Note that when communicating with READ\_HEAD, ROUTE does not need to send the tail bit; READ\_HEAD should never see a tail bit.

%%%%%%%%%%%%%%%%%%%%%%%%%%%%%%%%%%%%%%%%
\subsubsection*{CHP}

\begin{csp}
*[[#{C0!};C0!(P?)
  []#{H!};H!(P?)]
 ]
\end{csp}

%%%%%%%%%%%%%%%%%%%%%%%%%%%%%%%%%%%%%%%%
\subsubsection*{HSE}

% [[ci];pe+;[pi];co+;[~ci];pe-;[~pi];co-]
\begin{hse}
*[[c0e|he];pe+
    [ p0&c0e->c00+;[~c0e];pe-;[~p0];c00-
    []p1&c0e->c01+;[~c0e];pe-;[~p1];c01-
    []pt&c0e->c0t+;[~c0e];pe-;[~pt];c0t-
    []p0&he->h0+;[~he];pe-;[~p0];h0-
    []p1&he->h1+;[~he];pe-;[~p1];h1-
    ]
 ]
\end{hse}

%%%%%%%%%%%%%%%%%%%%%%%%%%%%%%%%%%%%%%%%
\subsubsection*{PRS}

\begin{prs2}
c0e | he -> pe+
~c0e & ~he -> pe-
\end{prs2}

\begin{prs2}
p0 & c0e -> c00+
~p0 -> c00-

p1 & c0e -> c01+
~p1 -> c01-

pt & c0e -> c0t+
~pt -> c0t-

p0 & he -> h0+
~p0 -> h0-

p1 & he -> h1+
~p1 -> h1-
\end{prs2}

\noindent Radix 2 accounting:

\begin{center}
    \begin{tabular}{|r|l|l|}
    \hline
    rule & transistor count & comments \\ \hline
    $pe$ & 4 & \\ \hline
    $c0[0,1,t]$ & 21 & \\ \hline
    $h[0,1]$ & 14 & \\ \hline
    \hline total & 39 & \\ \hline
    \end{tabular}
\end{center}

\noindent Radix 4 transistor accounting: 

\begin{center}
    \begin{tabular}{|r|l|l|}
    \hline
    rule & transistor count & comments \\ \hline
    $pe$ & 8 & \\ \hline
    $c0[0,1,2,3,t]$ & 35 & \\ \hline
    $h[0,1,2,3]$ & 28 & \\ \hline
    \hline total & 71 & \\ \hline
    \end{tabular}
\end{center}

%%%%%%%%%%%%%%%%%%%%%%%%%%%%%%%%%%%%%%%%%%%%%%%%%%%%%%%%%%%%%%%%%%%%%%%%%%%%%%%
\section{AERV PSAR RHB READ\_HEAD \label{sec:AERV_PSAR_RHB_READ_HEAD}}

%%%%%%%%%%%%%%%%%%%%%%%%%%%%%%%%%%%%%%%%
\subsubsection*{CHP}

%%%%%%%%%%%%%%%%%%%%%%%%%%%%%%%%%%%%%%%%
\subsubsection*{HSE}

% [[si];xe+;[xi];u+;xe-;[~xi];so+;[~si];u-;so-]
\begin{hse}
*[[si];xe+;
    [ x0->u0+;xe-;[~x0];s0+;[~si];u0-;s0-
    []x1->u1+;xe-;[~x1];s1+;[~si];u1-;s1-
    ]
 ]
\end{hse}

%%%%%%%%%%%%%%%%%%%%%%%%%%%%%%%%%%%%%%%%
\subsubsection*{PRS}

\begin{prs2}
si & ~u0 & ~u1 -> xe+
u0 | u1 -> xe-
\end{prs2}

\begin{prs2}
x0 -> u0+
~si -> u0-

x1 -> u1+
~si -> u1-

u0 & ~x0 -> s0+
~u0 | x0 -> s0-

u1 & ~x1 -> s1+
~u1 | x1 -> s1-
\end{prs2}

\noindent Radix 2 accounting:

\begin{center}
    \begin{tabular}{|r|l|l|}
    \hline
    rule & transistor count & comments \\ \hline
    $xe$ & 9 & \\ \hline
    $u[0,1]$ & 12 & \\ \hline
    $s[0,1]$ & 8 & \\ \hline
    \hline total & 29 & \\ \hline
    \end{tabular}
\end{center}

\noindent Radix 4 transistor accounting: 

\begin{center}
    \begin{tabular}{|r|l|l|}
    \hline
    rule & transistor count & comments \\ \hline
    $xe$ & 13 & \\ \hline
    $u[0,1,2,3]$ & 24 & \\ \hline
    $s[0,1,2,3]$ & 16 & \\ \hline
    \hline total & 53 & \\ \hline
    \end{tabular}
\end{center}

%%%%%%%%%%%%%%%%%%%%%%%%%%%%%%%%%%%%%%%%%%%%%%%%%%%%%%%%%%%%%%%%%%%%%%%%%%%%%%%
\section{AERV PSAR RHB FWD\_BODY unpipelined \label{sec:AERV_PSAR_RHB_FWD_BODY_u}}

%%%%%%%%%%%%%%%%%%%%%%%%%%%%%%%%%%%%%%%%
\subsubsection*{CHP}

%%%%%%%%%%%%%%%%%%%%%%%%%%%%%%%%%%%%%%%%
\subsubsection*{HSE}

% [so+;[si&ci];pe+;[pi];co+;[~ci];pe-;[~pe];co-]
\begin{hse}
*[[~s0&~s1->so+;
  []s0->[c0e];pe+;
    [ p0->c00+;[~c0e];pe-;[~p0];c00-
    []p1->c01+;[~c0e];pe-;[~p1];c01-
    []pt->c0t+;[~c0e];pe-;[~pt];so-;[~s0];c0t-
    ]
  []s1->[c1e];pe+;
    [ p0->c10+;[~c1e];pe-;[~p0];c10-
    []p1->c11+;[~c1e];pe-;[~p1];c11-
    []pt->c1t+;[~c1e];pe-;[~pt];so-;[~s1];c1t-
    ]
   ]
 ]
\end{hse}

%%%%%%%%%%%%%%%%%%%%%%%%%%%%%%%%%%%%%%%%
\subsubsection*{PRS}

\begin{prs2}
s0 & c0e | s1 & c1e -> pe+
s0 & ~c0e | s1 & ~c1e -> pe-

~c0t & ~c1t -> so+
~pt & (c0t | c1t) -> so-
\end{prs2}

\begin{prs2}
p0 & s0 -> c00+
~p0 | ~s0 -> c00-

p1 & s0 -> c01+
~p1 | ~s0 -> c01-

pt & s0 -> c0t+
~s0 -> c0t-

p0 & s1 -> c10+
~p0 | ~s1 -> c10-

p1 & s1 -> c11+
~p1 | ~s1 -> c11-

pt & s1 -> c1t+
~s1 -> c1t-
\end{prs2}

\noindent Radix 2 accounting:

\begin{center}
    \begin{tabular}{|r|l|l|}
    \hline
    rule & transistor count & comments \\ \hline
    $pe$ & 12 & \\ \hline
    $s_o$ & 9 & \\ \hline
    $c[0,1][0,1]$ & 16 & \\ \hline
    $c[0,1]t$ & 14 & \\ \hline
    \hline total & 51 & \\ \hline
    \end{tabular}
\end{center}

\noindent Radix 4 transistor accounting: 

\begin{center}
    \begin{tabular}{|r|l|l|}
    \hline
    rule & transistor count & comments \\ \hline
    $pe$ & 20 & \\ \hline
    $s_o$ & 13 & \\ \hline
    $c[0,1,2,3][0,1,2,3]$ & 64 & \\ \hline
    $c[0,1,2,3]t$ & 28 & \\ \hline
    \hline total & 125 & \\ \hline
    \end{tabular}
\end{center}

%%%%%%%%%%%%%%%%%%%%%%%%%%%%%%%%%%%%%%%%%%%%%%%%%%%%%%%%%%%%%%%%%%%%%%%%%%%%%%%
\section{AERV PSAR RHB FWD\_BODY pipelined \label{sec:AERV_PSAR_RHB_FWD_BODY_p}}

%%%%%%%%%%%%%%%%%%%%%%%%%%%%%%%%%%%%%%%%
\subsubsection*{CHP}

%%%%%%%%%%%%%%%%%%%%%%%%%%%%%%%%%%%%%%%%
\subsubsection*{HSE}

% [so+;[si];pe+;[pi&ci];co+;pe-;[~pe&~ci];co-]
\begin{hse}
*[so+;[s0|s1];pe+;
    [ p0&s0&c0e->c00+;pe-;[~p0&~c0e];c00-
    []p1&s0&c0e->c01+;pe-;[~p1&~c0e];c01-
    []pt&s0&c0e->c0t+;pe-;[~pt&~c0e];so-;[~s0];c0t-
    []p0&s1&c1e->c10+;pe-;[~p0&~c1e];c10-
    []p1&s1&c1e->c11+;pe-;[~p1&~c1e];c11-
    []pt&s1&c1e->c1t+;pe-;[~pt&~c1e];so-;[~s1];c1t-
    ]
 ]
\end{hse}

%%%%%%%%%%%%%%%%%%%%%%%%%%%%%%%%%%%%%%%%
\subsubsection*{PRS}

\begin{prs2}
s0 | s1 & ~q -> pe+
q -> pe-

~q -> so+
~pt & (~c0e & c0t | ~c1e & c1t)-> so-

c00 | c01 | c0t | c10 | c11 | c1t -> q+
~c00 & ~c01 & ~c0t & ~c10 & ~c11 & ~c1t -> q-
\end{prs2}

\begin{prs2}
p0 & s0 & c0e -> c00+
~p0 & ~c0e -> c00-

p1 & s0 & c0e -> c01+
~p1 & ~c0e -> c01-

pt & s0 & c0e -> c0t+
~s0 -> c0t-

p0 & s1 & c1e -> c10+
~p0 & ~c0e -> c10-

p1 & s1 & c1e -> c11+
~p1 & ~c0e -> c11-

pt & s1 & c1e -> c1t+
~s1 -> c1t-
\end{prs2}

\noindent Radix 2 accounting:

\begin{center}
    \begin{tabular}{|r|l|l|}
    \hline
    rule & transistor count & comments \\ \hline
    \hline total & 90 & \\ \hline
    \end{tabular}
\end{center}

\noindent Radix 4 transistor accounting: 

\begin{center}
    \begin{tabular}{|r|l|l|}
    \hline
    rule & transistor count & comments \\ \hline
    \hline total & 262 & \\ \hline
    \end{tabular}
\end{center}

%%%%%%%%%%%%%%%%%%%%%%%%%%%%%%%%%%%%%%%%%%%%%%%%%%%%%%%%%%%%%%%%%%%%%%%%%%%%%%%
\section{AERV PSAR decomposed into ROUTE PULL\_CTRL PULL (RCP) \label{sec:AERV_PSAR_RCP}}

ROUTE sends a parent's signal to one of its children depending on which child requests. Assumes requests are mutually exclusive. This is same ROUTE as above.

PULL\_CTRL reads the head word and indicates which PULL should request data from ROUTE.

PULL requests data from ROUTE and passes the data to the child. \\

\noindent Radix 2 accounting (4095 nodes / 4096 neurons):

\begin{center}
    \begin{tabular}{|r|l|l|l|}
    \hline
    component & transistors/component & components/node & transistors/node \\ \hline
    ROUTE & 62 & 1 & 62 \\ \hline
    PULL\_CTRL & 28 & 1 & 28 \\ \hline
    PULL & 14 & 2 & 28 \\ \hline
    \hline \multicolumn{3}{|r|}{total transistors/node} & 118 \\ \hline
    \end{tabular}
\end{center}

118 transistors/node * 4095 nodes / 4096 neurons = \textbf{118.0 transistors/neuron}

\noindent Radix 4 transistor accounting (1365 nodes / 4096 neurons): 

\begin{center}
    \begin{tabular}{|r|l|l|l|}
    \hline
    component & transistors/component & components/node & transistors/node \\ \hline
    ROUTE & 178 & 1 & 178 \\ \hline
    PULL\_CTRL & 58 & 1 & 58 \\ \hline
    PULL & 14 & 4 & 56 \\ \hline \hline
    \multicolumn{3}{|r|}{total transistors/node} & 292 \\ \hline
    \end{tabular}
\end{center}

292 transistors/node * 1365 nodes / 4096 neurons = \textbf{97.3 transistors/neuron}

However, we can still send a payload to the neurons with 1-of-2 data instead of 1-of-4 data at the leaf nodes.
There are 1024 leaf nodes. 
This will simplify the leaf node ROUTE component because its children only need to see the 1-bit payload and tail.
Each leaf node ROUTE can lose 1 bit (i.e. 2 data lines or 2 asymmetric c-elements or 14 transistors) for each of 4 children.
Therefore we can subtract 1024*14*4 = 57344 transistors

\noindent Leaving out the high bit from the leaf nodes yields

(292 transistors/node * 1365 nodes - 57344 transistors)/ 4096 neurons = \textbf{83.3 transistors/neuron}
%%%%%%%%%%%%%%%%%%%%%%%%%%%%%%%%%%%%%%%%%%%%%%%%%%%%%%%%%%%%%%%%%%%%%%%%%%%%%%%
\section{AERV PSAR RCP ROUTE \label{sec:AERV_PSAR_RCP_ROUTE}}

This decomposition largely reuses the unpipelined ROUTE in Section~\ref{sec:AERV_PSAR_RHB_ROUTE_u} above. 
For this decomposition, ROUTE connects to [radix] instances of PULL and 1 instance of PULL\_CTRL.

\noindent Radix 2 accounting:

\begin{center}
    \begin{tabular}{|r|l|l|}
    \hline
    rule & transistor count & comments \\ \hline
    $pe$ & 6 & \\ \hline
    $c[0,1][0,1,t]$ & 42 & \\ \hline
    $h[0,1]$ & 14 & \\ \hline
    \hline total & 62 & \\ \hline
    \end{tabular}
\end{center}

\noindent Radix 4 transistor accounting: 

\begin{center}
    \begin{tabular}{|r|l|l|}
    \hline
    rule & transistor count & comments \\ \hline
    $pe$ & 10 & \\ \hline
    $c[0,1,2,3][0,1,2,3,t]$ & 140 & \\ \hline
    $h[0,1,2,3]$ & 28 & \\ \hline
    \hline total & 178 & \\ \hline
    \end{tabular}
\end{center}

%%%%%%%%%%%%%%%%%%%%%%%%%%%%%%%%%%%%%%%%%%%%%%%%%%%%%%%%%%%%%%%%%%%%%%%%%%%%%%%
\section{AERV PSAR PULL\_CTRL \label{sec:AERV_PSAR_RCP_PULL_CTRL}}

%%%%%%%%%%%%%%%%%%%%%%%%%%%%%%%%%%%%%%%%
\subsubsection*{CHP}

%%%%%%%%%%%%%%%%%%%%%%%%%%%%%%%%%%%%%%%%
\subsubsection*{HSE}

\begin{hse}
*[[s0i&s1i];xe+;
  [ x0->u0+;xe-;[~x0];s0o+;[~s0i];u0-;s0o-
  []x1->u1+;xe-;[~x1];s1o+;[~s0i];u1-;s0o-
  ]]
\end{hse}

%%%%%%%%%%%%%%%%%%%%%%%%%%%%%%%%%%%%%%%%
\subsubsection*{PRS}

\begin{prs2}
s0i & s1i & q -> xe+
~s0i | ~s1i | ~q -> xe-

~u0 & ~u1 -> q+
u0 | u1 -> q-
\end{prs2}

\begin{prs2}
x0 -> u0+
~s0i -> u0-

x1 -> u1+
~s1i -> u1-
\end{prs2}

\begin{prs2}
u0 & ~x0 -> s0o+
~u0 | x0 -> s0o-

u1 & ~x1 -> s1o+
~u1 | x1-> s1o-
\end{prs2}

\noindent Radix 2 accounting:

\begin{center}
    \begin{tabular}{|r|l|l|}
    \hline
    rule & transistor count & comments \\ \hline
    $xe$ & 8 & no $q$ \\ \hline
    $u[0,1]$ & 12 & \\ \hline
    $s[0,1]_o$ & 8 & \\ \hline
    \hline total & 28 & \\ \hline
    \end{tabular}
\end{center}

\noindent Radix 4 transistor accounting: 

\begin{center}
    \begin{tabular}{|r|l|l|}
    \hline
    rule & transistor count & comments \\ \hline
    $xe$ & 10 & \\ \hline
    $q$ & 8 & \\ \hline
    $u[0,1,2,3]$ & 24 & \\ \hline
    $s[0,1,2,3]_o$ & 16 & \\ \hline
    \hline total & 58 & \\ \hline
    \end{tabular}
\end{center}

%%%%%%%%%%%%%%%%%%%%%%%%%%%%%%%%%%%%%%%%%%%%%%%%%%%%%%%%%%%%%%%%%%%%%%%%%%%%%%%
\section{AERV PSAR RCP PULL unpipelined \label{sec:AERV_PSAR_RCP_PULL_u}}

%%%%%%%%%%%%%%%%%%%%%%%%%%%%%%%%%%%%%%%%
\subsubsection*{CHP}

%%%%%%%%%%%%%%%%%%%%%%%%%%%%%%%%%%%%%%%%
\subsubsection*{HSE}

\begin{hse}
*[~si->so+;
 []si->[ye];xe+
  [ x0->y0+;[~ye];xe-;[~x0];y0-
  []x1->y1+;[~ye];xe-;[~x1];y1-
  []xt->yt+;[~ye];xe-;[~xt];so-;[~si];yt-
  ]
 ]
\end{hse}

%%%%%%%%%%%%%%%%%%%%%%%%%%%%%%%%%%%%%%%%
\subsubsection*{PRS}

\begin{prs2}
~yt & ~si -> so+
yt & ~xt -> so-

si & ye -> xe+
~si | ~ye -> xe-
\end{prs2}

\begin{prs2}
x0 -> y0+
~x0 -> y0-

x1 -> y1+
~x1 -> y1-

xt -> yt+
~si -> yt-
\end{prs2}

\noindent Radix 2 transistor accounting: 

\begin{center}
    \begin{tabular}{|r|l|l|}
    \hline
    rule & transistor count & comments \\ \hline
    $s_o$ & 4 & \\ \hline
    $xe$ & 4 & \\ \hline
    $y[0,1]$ & 0 & wires \\ \hline
    $yt$ & 6 & \\ \hline
    \hline total & 14 & \\ \hline
    \end{tabular}
\end{center}

\noindent Radix 4 transistor accounting: 

\begin{center}
    \begin{tabular}{|r|l|l|}
    \hline
    rule & transistor count & comments \\ \hline
    $s_o$ & 4 & \\ \hline
    $xe$ & 4 & \\ \hline
    $y[0,1,2,3]$ & 0 & wires \\ \hline
    $yt$ & 6 & \\ \hline
    \hline total & 14 & \\ \hline
    \end{tabular}
\end{center}

%%%%%%%%%%%%%%%%%%%%%%%%%%%%%%%%%%%%%%%%%%%%%%%%%%%%%%%%%%%%%%%%%%%%%%%%%%%%%%%
\end{document}
