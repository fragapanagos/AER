\documentclass{article}
\usepackage{mystyle}

\begin{document}
\title{Simple AER}
\author{Sam Fok}
\maketitle

In this document, we develop a simple address-event representation (AER) system for an array of $n$ neurons.

\begin{csp}
NT(n)\equiv\langle\pll\!i:0..n\-1:*[[P`i;P`i]]\rangle
NR(n)\equiv\langle\pll\!i:0..n\-1:*[[P`i;P`i]]\rangle
\end{csp}
%%%%%%%%%%%%%%%%%%%%%%%%%%%%%%%%%%%%%%%%%%%%%%%%%%%%%%%%%%%%%%%%%%%%%%%%%%%%%%%
\part{Transmitter ($AEXT$)}

When a neuron spikes, the transmitter sends the neuron's addresss.

%%%%%%%%%%%%%%%%%%%%%%%%%%%%%%%%%%%%%%%%%%%%%%%%%%%%%%%%%%%%%%%%%%%%%%%%%%%%%%%
\section{Transmitter Decomposition}

The highest level specification of the transmitter is given by

\begin{csp}
AEXT(n)\equiv
  *[[\langle\|\!i:0..n\-1:#{C`i}->T!enc(i);C`i;T!enc(i);C`i]]
\end{csp}

\noindent in words,

\begin{tabular}[]{rl}
  $\overline{C_i}$ & when the $i$th neuron spikes and is selected by the arbitration \\
  $T!enc(i)$ & encode and initiate neuron address transmission \\
  $C_i$ & signal the neuron's reset \\
  $T!enc(i)$ & complete neuron address transmission \\
  $C_i$ & release the neuron's reset \\
\end{tabular} \\

\noindent We decompose $AEXT$ into control $CTRL$ and data $DATA$ processes.

\begin{csp}
AEXT(n)\equiv*[CTRL(n)\pll\!DATA(n)]
\end{csp}

\begin{csp}
CTRL(n)\equiv*[[\langle\|i:0..n\-1:#{C`i}->C`i;C`i]]
\end{csp}

\noindent $CTRL$ is just an $n$-way arbiter, which is developed elsewhere.

\begin{csp}
DATA(n)\equiv\langle\pll\!i:0..n\-1:*[C`i;A`i;T!enc(i);T!enc(i);A`i;C`i]\rangle
\end{csp}

\noindent in words,

\begin{tabular}[]{rl}
  $C_i$ & wait for the $i$th child's request and signal child's reset \\
  $A_i$ & request the arbiter for permission to proceed \\
  $T!enc(i)$ & encode and initiate neuron address transmission \\
  $T!enc(i)$ & complete neuron address transmission \\
  $A_i$ & release arbiter \\
  $C_i$ & release child's reset \\
\end{tabular} \\ \\

\noindent $T$ is shared across the $n$ concurrent processes in $DATA$. We split out the transmition into a separate process.

\begin{csp}
DATA(n)\equiv\langle\pll\!i:0..n\-1:*[C`i;A`i;S`i;S`i;A`i;C`i]\rangle
\end{csp}

\begin{csp}
TX(n)\equiv*[[\langle[]S`i;T!enc(i);T!enc(i);S`i\rangle]]
\end{csp}

\noindent $NT$, $DATA$, $CTRL$, and $TX$ are connected as follows

\begin{csp}
\langle\,i:0..n\-1:
  DATA(n).C`i\Leftrightarrow\!NT(n).P`i\rangle

\langle\,i:0..n\-1:
  DATA(n).A`i\Leftrightarrow\!CTRL(n).C`i\rangle

\langle\,i:0..n\-1:
  DATA(n).S`i\Leftrightarrow\!TX(n).S`i\rangle
\end{csp}

\noindent Next, we pipeline and break symmetry (to eliminate indistinguishable states) in $DATA$ and $TX$.

\begin{csp}
DATA(n)\equiv\langle\pll\!i:0..n\-1:*[C`i;A`i;S`i;C`i\star(S`i,A`i)]\rangle
\end{csp}

\begin{csp}
TX(n)\equiv*[[\langle[]S`i\star\!T!enc(i);T!enc(i),S`i\rangle]]
\end{csp}

\subsection{HSE}

\begin{hse}
DATA\equiv*
  [[ci];co+,ao+;[ai];so+;[si];
   [~ci];so-,ao-;[~si&~ai];co-]

TX\equiv*
  [[si];to+;[ti];so+;
   to-;[~si];so-;[~ti]]
\end{hse}


%%%%%%%%%%%%%%%%%%%%%%%%%%%%%%%%%%%%%%%%%%%%%%%%%%%%%%%%%%%%%%%%%%%%%%%%%%%%%%%
\part{Receiver ($AERV$)}

%%%%%%%%%%%%%%%%%%%%%%%%%%%%%%%%%%%%%%%%%%%%%%%%%%%%%%%%%%%%%%%%%%%%%%%%%%%%%%%
\section{Receiver Decomposition}

The highest level of the receiver is given by

\begin{csp}
AERV(n)\equiv*[[\langle[]i:0..n\-1:C`i;C`i]]
\end{csp}
\end{document}
