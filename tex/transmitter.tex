\documentclass[aer.tex]{subfiles}
\begin{document}

\section{Transmitter}
The transmitter detects spike events and transmits
them out of the core (e.g. to a router or a receiver) in serialized AER
packets. Neurons are arranged in a 2D grid of rows (y address) and columns 
(x address). 
When a neuron spikes, it (and any other neuron in the same row) raises a 
request to the arbiter at the end of the rows. 
The row arbiter selects a row of neurons to transmit. The 
y address is then transmitted. Any neuron that spiked in the 
selected row then send their column addresses to the column latch and arbiter.
Their column addresses are latched to allow the neurons to reset while the 
column arbiter sequences the transmission of their x addresses. Once all of the
x addresses have been transmitted, the tail bit is transmitted to signal the 
end of the packet.

\subsection{Transmitter Decomposition}
Here we will start from a high level description of the transmitter and then decompose
the description into a set of concurrent, parallel processes. 
The highest level description of the address-event transmitter $AEXT$ is

\begin{csp}
AEXT(n)\equiv*[[\langle\|\,j:1..n:#{P`j}->A!enc(j)\pll\,P`j\rangle]]
\end{csp}
where 

\begin{tabular}[]{rl}
$n$ & is the number of neurons \\
$P_j$ & is the port communicating with the $j$th neuron \\
$A$ & is the output port \\
$enc(\cdot)$ & is a function returning the address of a neuron \\
\end{tabular} \\

In words, when the $j$th neuron spikes, output its address and reset the neuron. 
We rely on the "$\vert$" arbitration operator to handle multiple neurons spiking simultaneously.
We first expand $AEXT$ into rows and columns.

\begin{csp}
AEXT(x,y)\equiv*[[\langle\|\,l:1..y:\langle|k:1..x:#{P`{l\cdot\,k}}\rangle->A`1!enc(l)
               ,[\langle\|\,k:1..x:#{P`{l\cdot\,k}}->A`2!enc(k),P`{l\cdot\,k}\rangle]\rangle]]
\end{csp}
where 

\begin{tabular}[]{rl}
$x$ & is the number of columns \\
$y$ & is the number of rows \\
$P_{l\cdot k}$ & is the port communicating with the $l\cdot\,k$ neuron, and $\l\cdot k=xl+k$  \\
$A_1$ & is the row address output port \\
$A_2$ & is the column address output port \\
\end{tabular} \\

In words, for every row, wait for any neuron in that row to spike.
When a neuron spikes within a row, transmit the row address. In parallel,
also transmit the spiked neurons' column addresses while resetting the spiked neurons.

We still rely on the arbitration operator to handle neurons in multiple rows spiking concurrently, 
so our next step is to decompose $AEXT$ into separate processes for 
arbitration ($ARB$) and concurrent transmition ($ARY$). The arbitration process is given by

\begin{csp}
ARB(m)\equiv*[[\langle\|j:1..m:#{L`j}->L`j;L`j\rangle]]
\end{csp}
where $m$ is the number of processes to arbitrate and $L_j$ is port communicating with the $j$th process.
The first $L_j$ communication informs the $j$th process it may proceed, 
and the second $L_j$ communication confirms when the $j$th process has finished 
before the arbiter moves on to the next process. 
With the arbiter defined, we define the concurrent transmition process

\begin{csp}
ARY(y,x)\equiv*[[\langle,l:1..y:\langle|k:1..x:#{P`{l\cdot\,k}}\rangle->R`l;A`1!enc(l)
             ,[\langle,k:1..x:#{P`{l\cdot\,k}}->C`k;A`2!enc(k),P`{l\cdot\,k};C`k\rangle];R`l\rangle]]
\end{csp}
where

\begin{tabular}[]{rl}
$R_l$ & is the port communicating with the row arbiter \\
$C_k$ & is the port communicating with the column arbiter \\
\end{tabular} \\

From $AEXT$ to $ARY$, we replace arbitration "$\vert$" with concurrancy "$,$" and
introduce communications with the row and column arbiters.
As $ARB$ has two communications with each client: one to grant permission and one to check for 
completion, $ARY$ has two communications with the each arbiter: one to request permission and
one to indicate completion.

Next, we split out the address encoding function of $ARY$ into its own process.

\begin{csp}
ENC(m)\equiv*[[\langle[]j:1..m:#{L`j}->A!enc(j),L`j\rangle]]
\end{csp}

where $L$ communicates with $ARY$. We can use selection, which assumes mutual exclusion, 
in the encoding process because the arbiters ensure that we only attempt to encode one address at time.
Introducing a row and column address encoder, we rewrite $ARY$ as

\begin{csp}
ARY(y,x)\equiv*[[\langle,l:1..y:\langle|k:1..x:#{P`{l\cdot\,k}}\rangle->R`l;A`l
             ,[\langle,k:1..x:#{P`{l\cdot\,k}}->C`k;D`k,P`{l\cdot\,k};C`k\rangle];R`l\rangle]]
\end{csp}

where the $A_l$ port communicates with the row address encoder
and the $D_k$ port communicates with the column address encoder.

Next, we decompose $ARY$ into concurrent row and column processes.

\begin{csp}
ROW(x)\equiv*[[\langle|k:1..x:#{P`k}\rangle->R;
            A,[\langle,k:1..x:#{P`k}->C`k,P`k\rangle];R]]
COL(y)\equiv*[[\langle[]l:1..y:#{R`l}->C;D,R`l;C\rangle]]
\end{csp}

where

\begin{tabular}[c]{rl}
$P_k$ & communicates with the $k$th neuron in a row \\
$R$ & communicates the the row arbiter \\
$A$ & communicates with the row address encoder \\
$C_k$ & communicates with the $k$th column process \\
$R_l$ & communicates with the $l$th row process \\
$C$ & communicates with the column arbiter \\
$D$ & communicates with the column address encoder \\
\end{tabular}

In words, we have decomposed $ARY$ into $y$ $ROW$ processes and $x$ $COL$ processes. 
When a neuron spikes, its $ROW$ process makes a request to the row arbiter.
Once granted permission, $ROW$, in parallel, communicates with the row address encoder, 
signals the $COL$ processes corresponding to spiked neurons in the row to proceed, 
and sends reset signals to the spiked neurons.
When the row addess encoder, $COL$ processes, and neuron resets have completed, 
$ROW$ signals the row arbiter that it has finished.
The $COL$ processes use selection between the $R_l$ ports in the column because the 
row arbiter has ensured that only one $ROW$ process will signal the $COL$ process
at a time. 

Once signaled by a $ROW$ process, $COL$ makes a request to the column arbiter.
After gaining permission, $COL$ communicates with the column address encoder while
acknowledging the $ROW$ process. Finally, $COL$ signals the column arbiter that 
it has finished.

Our final step is to read out the activated columns in a row in parallel
and transmit the row and column addresses sequentially.
To read out columns in parallel, we modify $ROW$ and convert $COL$ into $LTH$ and $BUS$.
To sequence the address transmitions, we create $SEQ$.

$ROW$ is modified as

\begin{csp}
ROW(x)\equiv*[[\langle,k:1..x:#{P`k}->w.k+\rangle];R;
           A,C!w,[\langle,k:1..x:w.k->P`k\rangle];R,w:=0]
\end{csp}

where

\begin{tabular}[c]{rl}
$w$ & is an $x$-bit integer indicating which neurons spiked \\
$C$ & interfaces with $BUS$ (see below) \\
\end{tabular}

Note how $w$ is cleared at the end of the process as well. 
$BUS$ and $LTH$ are defined as

\begin{csp}
BUS(y,x)\equiv*[[\langle[]l:1..y:#{R`l}->C!(R`l?)\rangle]]
LTH(x)\equiv*[R?w;[\langle,k:1..x:w.k->C`k;D`k,w.k-;C`k\rangle];R]
\end{csp}

where for $BUS$,

\begin{tabular}[c]{rl}
$R_l$ & interfaces with an instance of $ROW$ \\
$C$ & interfaces with $SEQ$ (see below) \\
\end{tabular}

and for $LTH$

\begin{tabular}[c]{rl}
$R_l$ & interfaces with an instance of $ROW$ \\
$C$ & interfaces with $SEQ$ (see below) \\
$D$ & interfaces with an instance of $ENC$ \\
\end{tabular}

$BUS$ simply communicates data between $ROW$ and $SEQ$. 
$LTH$ reads in data from $SEQ$ and sends the column addresses to be encoded.
The second $R$ communication indicates that all column addresses have been encoded.

$SEQ$ sequences the events

\begin{csp}
SEQ(b,x)\equiv*[S!(D?)\pll\,T!(R?);S;T!\phi\]\pll*[T!(C?)]
\end{csp}

where 

\begin{tabular}[c]{rl}
$D$ & reads in data from $BUS$ \\
$S$ & interfaces with $LTH$ \\
$R$ & interfaces with the row $ENC$ \\
$C$ & interfaces with the column $ENC$ \\
$T$ & is the transmitter output port \\
$\phi$ & is a reserved keyword for a tailword \\
\end{tabular}

The second $S$ communication corresponds to $LTH$'s second $R$ communication and
indiciates that the latch is empty and that the transmitter may move on to the next row.
There is a timing assumption that row data arrives at $T$ before column data.
This assumption is valid because the row data has fewer processes to traverse
than column data before reaching $T$.

Now that we have decomposed the transmitter, we will proceed to HSE and PRS
for the decomposed components.

%%%%%%%%%%%%%%%%%%%%%%%%%%%%%%%%%%%%%%%%%%%%%%%%%%%%%%%%%%%%%%%%%%%%%%%%%%%%%%%
\subsection{Event Generator (EVT)}
EVT interfaces the neuron with the rest of the transmitter. 
There is one EVT per neuron.
EVT has the following ports:

\begin{tabular}[]{rll}
  \code{P} & (passive) & communicates with the neuron \\
  \code{R} & (active) & communicates with row interface INT \\
  \code{C} & (active) & communicates with the column latch LTH \\
\end{tabular} \\ \\

EVT's HSE is given by

\begin{hse}
*[[pi];ro+;[ri];co+,po+;
 [~pi];ro-;[~ri];co-,po-]
\end{hse}  

In words,

\begin{tabular}[]{rl}
  \code{[pi]} & wait for the neuron to spike \\
  \code{ro$\uparrow$;[ri]} & get permission from the arbiter interface / row arbiter to go ahead \\
  \code{co$\uparrow$,po$\uparrow$} & send the column signal and the neuron reset signal \\
  \code{[$\neg$pi]} & wait for the neuron to reset \\
  \code{ro$\downarrow$;[$\neg$ri]} & indicate that the row arbiter can select the next row. \\
  \code{co$\downarrow$, po$\downarrow$} & reset the column request and neuron reset signals \\
\end{tabular} \\ \\

Note that the \code{[$\neg$ri]} wait also indicates that the sequencer is done with this row 
(and acknowledges \code{co$\uparrow$}). EVT's PRS is given by 

\begin{prs2}
 pi&~ri->ro+
~pi->ro-

 ri&ro->co+,po+
~ri->co-,po-
\end{prs2}
The \code{ro$\uparrow$} guard is strengthened with 
\code{$\neg$ri} so that if the neuron spikes immediately after it has
reset, \code{[$\neg$pi]}, 
we do not try to raise \code{ro} while executing \code{ro$\downarrow$}. 
The \code{co$\uparrow$,po$\uparrow$} guard is strengthened with \code{ro} because \code{ri} is a wire connecting all EVT instances in a row. We only want the neurons that have actually spiked
to proceed with raising their column and neuron reset lines.

\emph{The \code{co$\uparrow$} transition is not explicitely acknowledged. 
I hope this is accounted for by \code{$\neg$ri}...Check this after implementing LTH.}

Next, we explicitly represent the isochronic fork between
\code{co} and \code{ro} using a local variable \code{u}.
To make the PRS CMOS-implementable, we invert the sense of \code{pi}, \code{co}, and \code{po}.

\begin{prs2}
 ~_pi & ~ri -> ro+
_pi         -> ro-

u  -> _co-
~u -> _co+

 ri & ro -> u-
~ri      -> u+

u  -> _po-
~u -> _po+
\end{prs2}

Initially, \code{$\neg$po$\land\neg$ro$\land\neg$co} is true. With reset circuitry,

\begin{prs2}
~sReset & ~_pi & ~ri -> ro+
pReset | _pi -> ro-

u  -> _co-
~u -> _co+

_sReset & ri & ro -> u-
~_pReset | ~ri -> u+

u  -> _po-
~u -> _po+
\end{prs2}

%%%%%%%%%%%%%%%%%%%%%%%%%%%%%%%%%%%%%%%%%%%%%%%%%%%%%%%%%%%%%%%%%%%%%%%%%%%%%%%
\subsection{Interface (INT)}

There are two sets of INTs in the transmitter:
one for the rows and another for the columns.
The row INTs communicate with 
EVT, the row arbiter, and the y-address controller (ADY).
The column INTs communicate with 
the column address latch (LTH), the column arbiter, and the x-address controller (ADX).
Each row (column) INT has the following ports:

\begin{tabular}[]{rll}
  \code{V} & (passive) & receives requests from EVT (LTH) \\
  \code{C} & (active) & transmits request to the row (column) arbiter \\
  \code{E} & (active) & communicates with the row (column) address controller \\
\end{tabular} \\ \\

For row INTs, \code{vi} is the wired-or output of all EVT \code{ro} ports in the row. Likewise,
\code{vo} is connected to all of the EVT \code{ri} ports in the row.

INT's HSE is given by

\begin{hse}
*[[vi];co+;[ci&~ei];eo+,vo+;
 [~vi&ei];co-;[~ci];eo-,vo-]
\end{hse}

In words,

\begin{tabular}[]{rl}
  \code{[vi]} & wait for a request from INT or LTH for the arbiter \\
  \code{co$\uparrow$} & send request to arbiter \\
  \code{[ci$\land\neg$ei]} & wait for arbiter to grant permission and address controller to finish previous transmission \\
  \code{eo$\uparrow$, vo$\uparrow$} & \specialcell[t]{l}{
    signal address latch to store address \\ 
    signal EVT to transmit column information} \\
  \code{[$\neg$vi$\land$ei]} & \specialcell[t]{l}{
    wait for neuron to reset or LTH to clear \\
    and address controller to have latched the address} \\
  \code{co$\downarrow$;[$\neg$ci]} & release arbiter \\
  \code{eo$\downarrow$, vo$\downarrow$} & reset signals to address latch and EVT/LTH \\
\end{tabular} \\ \\

Note how INT signals the address encoder first after gaining permission from the arbiter
and second after the arbiter has been released instead of before the arbiter has been released.
This is safe because INT waits until \code{ei} clears before the first \code{E} communication. 
That is, \code{E} follows a lazy-active protocol.
INT's PRS is given by

\begin{prs2}
 vi      -> co+
~vi & ei -> co-

 ci & ~ei -> eo+, vo+
~ci       -> eo-, vo-
\end{prs2}

Next, we represent the isochronic fork between \code{eo} and \code{vo}
using local variable \code{y}. 
To make the PRS CMOS-implementable, we introduce local variable \code{x} and
invert the sense of \code{vi} and \code{ei}.

\begin{prs2}
vi         -> x-
~vi & ~_ei -> x+

ci & _ei -> y-
~ci      -> y+

y  -> eo-
~y -> eo+

~x -> co+
x  -> co-

y  -> vo-
~y -> vo+
\end{prs2}

Initially, \code{$\neg$co$\land\neg$vo$\land\neg$eo} is true. With reset circuitry,

\begin{prs2}
_sReset & vi          -> x-
~_pReset | ~vi & ~_ei -> x+

_sReset & ci & _ei -> y-
~_pReset | ~ci     -> y+

y -> eo-
~y -> eo+

~x -> co+
x -> co-

y -> vo-
~y -> vo+
\end{prs2}

%%%%%%%%%%%%%%%%%%%%%%%%%%%%%%%%%%%%%%%%%%%%%%%%%%%%%%%%%%%%%%%%%%%%%%%%%%%%%%%
\subsection{Arbiter (ARB)}

ARB sequences requests from rows or columns by arbitrating between 
concurrent requests. For $N$ request lines, the arbiter is composed of $N-1$ 
arbiter circuits arranged in a binary tree. The arbiter has ports:

\begin{tabular}[]{rll}
  \code{R} & (active) & transmits request to parent \\
  \code{L1} & (passive) & receives requests from left child 1 \\
  \code{A1} & (local) & ? \\
  \code{R1} & (local) & ? \\
  \code{L2} & (passive) & receives requests from left child 2 \\
  \code{A2} & (local) & ? \\
  \code{R2} & (local) & ? \\
\end{tabular} \\ \\

\begin{hse}
*[[l1i&~ri];a1o+;r1o+;[ri&a1i];l1o+;
  [~l1i];a1o-;r1o-;[~a1i];l1o-]

*[[l2i&~ri];a2o+;r2o+;[ri&a2i];l2o+;
  [~l2i];a2o-;r2o-;[~a2i];l2o-]
\end{hse}

I will consider the prs for the arbiter later
%%%%%%%%%%%%%%%%%%%%%%%%%%%%%%%%%%%%%%%%%%%%%%%%%%%%%%%%%%%%%%%%%%%%%%%%%%%%%%%
\subsection{Latch (LTH)}

LTH is responsible for storing the column address of a neuron. 
\begin{itemize}
\item \code{R} (passive) communication line with sequencer
\item \code{Rk} (passive) column request from EVT
\item \code{G} (active) request line to column interface (to the arbiter)
\end{itemize}

\begin{hse}
*[[ri&rk];go+;ro+;[~ri&gi];go-;[~gi];ro-]
\end{hse}

\begin{prs2}
 ri&rk->go+
~ri&gi->go-

 gi|go->ro+
~gi&~go->ro-
\end{prs2}

Note how we made the guard for \code{ro} combinational by using \code{gi} in
the pullup.

Questions:
\begin{itemize}
\item What is the open dot feeding into \code{rk}?
\item What prevents \code{rk} go low before \code{go} completes? 
How does sequencer enforce this?
\end{itemize}

%%%%%%%%%%%%%%%%%%%%%%%%%%%%%%%%%%%%%%%%%%%%%%%%%%%%%%%%%%%%%%%%%%%%%%%%%%%%%%%
\subsection{Y Address Encoder (ADY)}

ADY is responsible for controlling the storage of y-addresses.

\begin{itemize}
\item \code{D}
\item \code{Y}
\item \code{V}
\end{itemize}

\begin{hse}
*[[yi];yo+;[vy];do+;[~yi];yo-;[~vy&~di];do-]
\end{hse}

\begin{prs}
\end{prs}

\section{Test structures}
Here we describe some processes used to test the transmitter processes

\subsection{EVT}
For testing evt, I used a dummy LTH\_INT process that takes the place of the LTH and INT blocks.
LTH\_INT's HSE is given by

\begin{hse}
*[[vi&~rk];vo+;[~vi&rk];vo-]
\end{hse}

The LTH\_INT's PRS is given by 

\begin{prs}
vi&~rk -> vo+
~vi&rk -> vo-
\end{prs}

\end{document}
