\documentclass{article}
\usepackage{mystyle}

\begin{document}
\title{Tree AER}
\author{Sam Fok}
\maketitle

%%%%%%%%%%%%%%%%%%%%%%%%%%%%%%%%%%%%%%%%%%%%%%%%%%%%%%%%%%%%%%%%%%%%%%%%%%%%%%%
\section{Transmitter ($AEXT$)}

Each node of the tree has two passive input ports, $D0$ and $D1$, and one active output port, $X$.

$D0$ and $D1$ are merged into $X$. Arbitration occurs between the $D0$ and $D1$ requests. 

\begin{csp}
*[[#{D0}->D0\*X!(0,D0?)
  \|#{D1}->D1\*X!(1,D1?)]]
\end{csp}

Where the bullet operator indicates the there will be some interleaving between the two communications. Which communication ordering hasn't been settled at this moment. All versions use the standard arbiter given by

\begin{hse}
*[[a0i->a0+;[~a0i];a0-
  \|a1i->a1+;[~a1i];a1-]]
\end{hse}

The intermediate nodes generate their arbiter requests using completion circuits:

\begin{hse}
*[[V(D0)];a0i+;[N(D0)];a0i-]
*[[V(D1)];a1i+;[N(D1)];a1i-]
\end{hse}

%%%%%%%%%%%%%%%%%%%%%%%%%%%%%%%%%%%
\subsection{Transmitter ordering $D\!\star\!X;D\!\star\!X$}

%%%%%%%%%%%%%%%%
\subsubsection{Leaf node}

\begin{csp}
*[[D0\star\!X!0;D0\star\!X
  \|D1\star\!X!1;D1\star\!X]]
\end{csp}

\begin{hse}
*[[a0->x`{00}+;[xi];d0o+;[~a0];x`{00}-;[~xi];d0o-
  []a1->x`{01}+;[xi];d1o+;[~a1];x`{01}-;[~xi];d1o-]]
\end{hse}

%%%%%%%%%%%%%%%%
\subsubsection{Intermediate nodes}

\begin{csp}
*[[D0\star\!X!(0,D0?);D0\star\!X
  \|D1\star\!X!(1,D1?);D1\star\!X]]
\end{csp}

\begin{hse}
*[[a0->x`{00}+,\langle,m:M:[d0`{m0}->x`{(m\+1)0}+[]d0`{m1}->x`{(m\+1)1}+]\rangle;[xi];d0o+;
  [~a0];x\!\Downarrow;[~xi];d0o-
  []a1->x`{01}+,\langle,m:M:[d1`{m0}->x`{(m\+1)0}+[]d1`{m1}->x`{(m\+1)1}+]\rangle;[xi];d1o+;
  [~a1];x\!\Downarrow;[~xi];d1o-
 ]]
\end{hse}

%%%%%%%%%%%%%%%%
\subsubsection{PRS}

\noindent \textbf{Leaf and intermediate nodes...creating the highest order bit}

\begin{prs2}
a0 & ~d1o -> x`{00}+
~a0 | d1o -> x`{00}-

a1 & ~d0o -> x`{01}+
~a1 | d0o -> x`{01}-
\end{prs2}

\noindent \textbf{Intermediate nodes only...transmitting the existing lower order bits}

\begin{prs2}
(a0 & d0`{m0} | a1 & d1`{m0}) & ~d0o & ~d1o -> x`{(m\+1)0}+

~a0 & d0o | ~a1 & d1o -> x`{(m\+1)0}-
\end{prs2}

\begin{prs2}
(a0 & d0`{m1} | a1 & d1`{m1}) & ~d0o & ~d1o -> x`{(m\+1)1}+

~a0 & d0o | ~a1 & d1o -> x`{(m\+1)1}-
\end{prs2}

\noindent \textbf{Common to all nodes...}

\begin{prs2}
xi & a0 & ~d1o -> d0o+
~xi -> d0o-

xi & a1 & ~d0o -> d1o+
~xi -> d1o-
\end{prs2}

%%%%%%%%%%%%%%%%
\subsubsection{CMOS-implementable PRS}
We alternate between version 1 and 2 described below at each layer. Using only one of the versions would require more inverters. The leaves will use version 1, though this is arbitrary.

\noindent \\ \textbf{Version 0}

\noindent \textbf{Leaf and intermediate nodes...creating the highest order bit}

\begin{prs2}
~_a0 & ~d1o -> x`{00}+
_a0 | d1o -> x`{00}-

~_a1 & ~d0o -> x`{01}+
_a1 | d0o -> x`{01}-
\end{prs2}

\noindent \textbf{Intermediate nodes only...transmitting the existing lower order bits}

\begin{prs2}
(~_a0 & ~_d0`{m0} | ~_a1 & ~_d1`{m0}) & ~d0o & ~d1o -> x`{(m\+1)0}+

_a0 & d0o | _a1 & d1o -> x`{(m\+1)0}-
\end{prs2}

\begin{prs2}
(~_a0 & ~_d0`{m1} | ~_a1 & ~_d1`{m1}) & ~d0o & ~d1o -> x`{(m\+1)1}+

_a0 & d0o | _a1 & d1o -> x`{(m\+1)1}-
\end{prs2}

\noindent \textbf{Common to all nodes...}

\begin{prs2}
~_xi & ~_a0 & ~d1o -> d0o+
_xi -> d0o-

~_xi & ~_a1 & ~d0o -> d1o+
_xi -> d1o-
\end{prs2}

\noindent \\ \textbf{Version 1}

\noindent \textbf{Leaf and intermediate nodes...creating the highest order bit}

\begin{prs2}
a0 & _d1o -> _x`{00}-
~a0 | ~_d1o -> _x`{00}+

a1 & _d0o -> _x`{01}-
~a1 | ~_d0o -> _x`{01}+
\end{prs2}

\noindent \textbf{Intermediate nodes only...transmitting the existing lower order bits}

\begin{prs2}
(a0 & d0`{m0} | a1 & d1`{m0}) & _d0o & _d1o -> _x`{(m\+1)0}-

~a0 & ~_d0o | ~a1 & ~_d1o -> _x`{(m\+1)0}+
\end{prs2}

\begin{prs2}
(a0 & d0`{m1} | a1 & d1`{m1}) & _d0o & _d1o -> _x`{(m\+1)1}-

~a0 & ~_d0o | ~a1 & ~_d1o -> _x`{(m\+1)1}+
\end{prs2}

\noindent \textbf{Common to all nodes...}

\begin{prs2}
xi & a0 & _d1o -> _d0o-
~xi -> _d0o+

xi & a1 & _d0o -> _d1o-
~xi -> _d1o+
\end{prs2}

%%%%%%%%%%%%%%%%
\subsubsection{Accounting}

For 4096 neurons, the transmitter requires 335,172 transistors, or 82 transistors per neuron, or 16.4$\mu\textrm{m}^2$ per neuron (See Table~\ref{tab:aext_cost}).

\begin{table}
  \centering
  \begin{tabular}{|r|c|c|c|c|c|c|c|c|c|c|c|c|}
    \hline
    Level & 0 & 1 & 2 & 3 & 4 & 5 & 6 & 7 & 8 & 9 & 10 & 11 \\ \hline
    Nodes & 2048 & 1024 & 512 & 256 & 128 & 64 & 32 & 16 & 8 & 4 & 2 & 1 \\ \hline \hline
    \multicolumn{13}{|c|}{Transistors} \\ \hline
    VN & 0 & 12288 & 18432 & 15360 & 10752 & 6912 & 4224 & 2496 & 1440 & 816 & 456 & 252 \\ \hline
    Arbiter & 24576 & 12288 & 6144 & 3072 & 1536 & 768 & 384 & 192 & 96 & 48 & 24 & 12 \\ \hline
    Merge & 16384 & 36864 & 32768 & 23552 & 15360 & 9472 & 5632 & 3264 & 1856 & 1040 & 576 & 316 \\ \hline
    Total & 335172 & \multicolumn{11}{}{} \\ \cline{1-2}
    per neuron & 82 & \multicolumn{11}{}{} \\ \cline{1-2}
    Area per neuron & 16.4 $\mu\textrm{m}^2$ & \multicolumn{11}{}{} \\ \cline{1-2}
  \end{tabular}
  \caption{\label{tab:aext_cost}Transmitter requirements for 4096 neurons. Each node has 2 VN detectors. VN detector transistor requirements are in Table~\ref{tab:vn_cost}. A 2-input arbiter requires $12$ transistors. There are 4096 inverters (8192 transistors) at the interface with the neurons. Area calculation assumes 2$\mu\textrm{m}^2$ per 10 transistors in 28nm technology.}
\end{table}

\begin{table}
  \centering
  \begin{tabular}{|c|c|c|c|c|c|c|c|c|c|c|c|c|}
    \hline
    \multicolumn{12}{|c|}{VN detector scaling} \\
    \hline
    1-of-2 Groups & 1 & 2 & 3 & 4 & 5 & 6 & 7 & 8 & 9 & 10 & 11 \\
    \hline
    Gates & 1 & 3 & 5 & 7 & 9 & 11 & 13 & 15 & 17 & 19 & 21 \\
    \hline
    Transistors & 6 & 18 & 30 & 42 & 54 & 66 & 78 & 90 & 102 & 114 & 126 \\
    \hline
  \end{tabular}
  \caption{\label{tab:vn_cost} Cost of 1-of-2 VN detectors. For $M$ 1-of-2 groups, we use $M$ OR-gates followed by a tree of $M-1$ C-elements. Each OR-gate can be effectively implemented by a NOR-gate and inverting the output of the C-element tree.}
\end{table}

%%%%%%%%%%%%%%%%%%%%%%%%%%%%%%%%%%%%%%%%%%%%%%%%%%%%%%%%%%%%%%%%%%%%%%%%%%%%%%%
\section{Receiver ($AERV$)}

The receiver simply looks at the highest order bit and directs the traffic accordingly.

%%%%%%%%%%%%%%%%%%%%%%%%%%%%%%%%%%%
\subsection{Receiver ordering: $D\!\star\!X;D\!\star\!X$}

%%%%%%%%%%%%%%%%
\subsubsection{Leaf level}

\begin{csp}
*[[D=0->D\star\!X0[]D=1->D\star\!X1];D\star\!(X0,X1)]
\end{csp}

\begin{hse}
*[[d`{00}];x0o+;[x0i];do+;[~d`{00}];x0o-,x1o-;[~x0i&~x1i];do-]
*[[d`{01}];x1o+;[x1i];do+;[~d`{01}];x0o-,x1o-;[~x0i&~x1i];do-]
\end{hse}

%%%%%%%%%%%%%%%%
\subsubsection{Higher levels}

\begin{csp}
*[D?y\star\![y=0->X0!y[]y=1->X1!y];D\star\!(X0,X1)]
\end{csp}

\begin{hse}
*[[d`{00}];\langle,m:1..M\-1:[d`{m0}->x0`{(m\-1)0}+[]d`{m1}->x0`{(m\-1)1}+]\rangle;[x0i];do+;
  [~d`{00}];x0\!\Downarrow;[~x0i&~x1i];do-]
*[[d`{01}];\langle,m:1..M\-1:[d`{m0}->x1`{(m\-1)0}+[]d`{m1}->x1`{(m\-1)1}+]\rangle;[x1i];do+;
  [~d`{01}];x1\!\Downarrow;[~x0i&~x1i];do-]
\end{hse}

%%%%%%%%%%%%%%%%
\subsubsection{PRS}

\textbf{Leaf node...communicate with neurons}

\begin{prs2}
d`{00} -> x0o+
~d`{00} -> x0o-

d`{01} -> x1o+
~d`{01} -> x1o-
\end{prs2}

\noindent \textbf{Intermediate nodes...route lower order bits to next lower level}

\begin{prs2}
d`{00} & d`{m0} -> x0`{(m\-1)0}+
~d`{00} & ~d`{m0} -> x0`{(m\-1)0}-

d`{01} & d`{m0} -> x1`{(m\-1)0}+
~d`{01} & ~d`{m0} -> x1`{(m\-1)0}-
\end{prs2}

\begin{prs2}
d`{00} & d`{m1} -> x0`{(m\-1)1}+
~d`{00} & ~d`{m1} -> x0`{(m\-1)1}-

d`{01} & d`{m0} -> x1`{(m\-1)1}+
~d`{01} & ~d`{m0} -> x1`{(m\-1)1}-
\end{prs2}

\noindent \textbf{Common to all nodes...acknowledge data}

\begin{prs2}
x0i | x1i -> do+
~x0i & ~x1i -> do-
\end{prs2}

%%%%%%%%%%%%%%%%
\subsubsection{CMOS-implementable PRS}

\textbf{Leaf node...communicate with neurons}

\noindent Wires. No transistors required.

\begin{prs2}
d`{00} -> x0o+
~d`{00} -> x0o-

d`{01} -> x1o+
~d`{01} -> x1o-
\end{prs2}

\noindent \textbf{Intermediate nodes...route lower order bits to next lower level}

\begin{prs2}
d`{00} & d`{m0} -> _x0`{(m\-1)0}+
~d`{00} & ~d`{m0} -> _x0`{(m\-1)0}-

d`{01} & d`{m0} -> _x1`{(m\-1)0}+
~d`{01} & ~d`{m0} -> _x1`{(m\-1)0}-
\end{prs2}

\begin{prs2}
d`{00} & d`{m1} -> _x0`{(m\-1)1}+
~d`{00} & ~d`{m1} -> _x0`{(m\-1)1}-

d`{01} & d`{m0} -> _x1`{(m\-1)1}+
~d`{01} & ~d`{m0} -> _x1`{(m\-1)1}-
\end{prs2}


\begin{prs2}
_x0`{(m\-1)0} -> x0`{(m\-1)0}-
~_x0`{(m\-1)0} -> x0`{(m\-1)0}+

_x1`{(m\-1)0} -> x1`{(m\-1)0}-
~_x1`{(m\-1)0} -> x1`{(m\-1)0}+
\end{prs2}

\begin{prs2}
_x0`{(m\-1)1} -> x0`{(m\-1)1}-
~_x0`{(m\-1)1} -> x0`{(m\-1)1}+

_x1`{(m\-1)1} -> x1`{(m\-1)1}-
~_x1`{(m\-1)1} -> x1`{(m\-1)1}+
\end{prs2}

\noindent 4 C-elements per output bit = 32 transistors per output bit.

\noindent \textbf{Common to all nodes...acknowledge data}

\noindent We implement the OR gate by alternanting NAND and NOR gates at each level of the tree.

\noindent Starting at the bottom of the tree and at every other level:
\begin{prs2}
x0i | x1i -> _do+
~x0i & ~x1i -> _do-
\end{prs2}

\noindent Starting at one up from the bottom of the tree and at every other level:
\begin{prs2}
_x0i & _x1i -> do-
~_x0i | ~_x1i -> do+
\end{prs2}

\noindent This requires $4$ transistors per node. If there are an odd number of levels, we add an inverter to the top of the tree.

%%%%%%%%%%%%%%%%
\subsubsection{Accounting}

For 4096 neurons, the receiver requires 147,036 transistors, or 36 transistors per neuron, or 7.2$\mu\textrm{m}^2$ per neuron (See Table~\ref{tab:aerv_cost}).

\begin{table}
  \centering
  \begin{tabular}{|r|c|c|c|c|c|c|c|c|c|c|c|c|}
    \hline
    Level & 0 & 1 & 2 & 3 & 4 & 5 & 6 & 7 & 8 & 9 & 10 & 11 \\ \hline
    Nodes & 2048 & 1024 & 512 & 256 & 128 & 64 & 32 & 16 & 8 & 4 & 2 & 1 \\ \hline \hline
    \multicolumn{13}{|c|}{Transistors} \\ \hline
    Split & 0 & 32768 & 32768 & 24576 & 16384 & 10240 & 6144 & 3584 & 2048 & 1152 & 640 & 352 \\ \hline
    Ack & 8192 & 4096 & 2048 & 1024 & 512 & 256 & 128 & 64 & 32 & 16 & 8 & 4 \\ \hline  
    Total & 147036 & \multicolumn{11}{}{} \\ \cline{1-2}
    per neuron & 35.9 & \multicolumn{11}{}{} \\ \cline{1-2}
    Area per neuron & 7.2 $\mu\textrm{m}^2$ & \multicolumn{11}{}{} \\ \cline{1-2}
  \end{tabular}
  \caption{\label{tab:aerv_cost}Receiver requirements for 4096 neurons. Area calculation assumes 2$\mu\textrm{m}^2$ per 10 transistors in 28nm technology.}
\end{table}

%%%%%%%%%%%%%%%%%%%%%%%%%%%%%%%%%%%%%%%%%%%%%%%%%%%%%%%%%%%%%%%%%%%%%%%%%%%%%%%
\section{Appendix}

%%%%%%%%%%%%%%%%%%%%%%%%%%%%%%%%%%%
\subsection{Transmitter ordering: $D;X;X;D$}

I would prefer this ordering because it would allow the data to be pipelined, but the PRS may not be so simple...

%%%%%%%%%%%%%%%%%%%%%%%%%%%%%%%%%%%
\subsection{Transmitter ordering: $D\!\star\!X;X;D$}

%%%%%%%%%%%%%%%%
\subsubsection{First level}

\begin{csp}
*[[D0\star\!X!0;X;D0
  \|D1\star\!X!1;X;D1]]
\end{csp}

\begin{hse}
*[[a0->x`{00}+;[xi];d0o+;x`{00}-;[~xi&~a0];d0o-
  []a1->x`{01}+;[xi];d1o+;x`{01}-;[~xi&~a1];d1o-]]
\end{hse}

%%%%%%%%%%%%%%%%
\subsubsection{Subsequent levels}

\begin{csp}
*[[D0\star\!X!(0,D0?);X;D0
  \|D1\star\!X!(1,D1?);X;D1]]
\end{csp}

\begin{hse}
*[[a0->x`{00}+,\langle,m:M:[d0`{m0}->x`{(m\+1)0}+[]d0`{m1}->x`{(m\+1)1}+]\rangle;[xi];d0o+;
   x\!\Downarrow;[~xi&~a0];d0o-
  []a1->x`{01}+,\langle,m:M:[d1`{m0}->x`{(m\+1)0}+[]d1`{m1}->x`{(m\+1)1}+]\rangle;[xi];d1o+;
   x\!\Downarrow;[~xi&~a1];d1o-
 ]]
\end{hse}

%%%%%%%%%%%%%%%%
\subsubsection{PRS}

First and subsequent level...

\begin{prs2}
a0 & ~d1o & ~d0o-> x`{00}+
d0o -> x`{00}-

a1 & ~d0o & ~d1o -> x`{01}+
d1o -> x`{01}-
\end{prs2}

\noindent Subsequent levels only...

\begin{prs2}
((a0 & d0`{m0}) | (a1 & d1`{m0})) & ~d0o & ~d1o -> x`{(m\+1)0}+

d0o | d1o -> x`{(m\+1)0}-
\end{prs2}

\begin{prs2}
((a0 & d0`{m1}) | (a1 & d1`{m1})) & ~d0o & ~d1o -> x`{(m\+1)1}+

d0o | d1o -> x`{(m\+1)1}-
\end{prs2}

\noindent Common to all levels...

\begin{prs2}
xi & a0 & ~d1o -> d0o+
~xi & ~a0 -> d0o-

xi & a1 & ~d0o-> d1o+
~xi & ~a1 -> d1o-
\end{prs2}


%%%%%%%%%%%%%%%%%%%%%%%%%%%%%%%%%%%%%%%%%%%%%%%%%%%%%%%%%%%%%%%%%%%%%%%%%%%%%%%
\end{document}
